\section{Discusi\'{o}n de resultados}


Lo mas destacable de las pruebas realizadas son los resultados de los sujetos 
 \textbf{n\'umero 4} y \textbf{n\'umero 5}
 (ver tabla \ref{infodata}, \ref{tableRes1} y 
 \ref{tableRes2}) ya que pese a ser los que menos tiempo de uso y
 \emph{Secuencias Totales} obtuvieron, su \emph{Porcentaje de Precisi\'on}
 no var\'ia tanto con respecto al de los dem\'as sujetos, lo cual demuestra 
 la estabilidad del algoritmo mostrado, porque, considerando los datos de la 
 tabla \ref{infodata}, el sujeto \textbf{n\'umero 4} realiz\'o mas 
 repeticiones que el sujeto \textbf{n\'umero 1} de la reducida cantidad de 
 acciones realizadas, es decir, el sujeto \textbf{n\'umero 4} realiz\'o 
 acciones similares cada vez que usaba la computadora, mientras que los 
 dem\'as sujetos tuvieron m\'as diversidad en las acciones realizadas.
 
 
Considerando que los criterios mencionados son para darle un uso espec\'ifico
 al grafo, es destacable que pese al entorno variable en el cual fue evaluado
 el software, solo fue necesario asignar unas pocas
 condiciones generales para obtener las secuencias coherentes. 
 Entre las secuencias rechazadas se pueden localizar tareas que tienen la
 longitud de una acci\'on, las cuales es posible que para alg\'un usuario 
 sean de utilidad, se presenta la siguiente lista a modo de ejemplo.
\\
\\
Secuencia 1:\\
Mouse,Scrolled,Down\\
\\
Secuencia 2:\\
Mouse,Scrolled,Up\\

Cabe recordar que la principal intenci\'on de este software es el apoyar a 
 las personas con discapacidad en brazos y manos, teniendo en cuenta esto, 
 las secuencias de longitud uno, pueden ser de utilidad a alguna persona, sin 
 embargo, considerando que la secuencia m\'as \'util es la que contiene mas 
 elementos, estas acciones unitarias fueron descartadas. Adicionalmente, si 
 se piensa en alg\'un otro uso para el algoritmo, aparte de la 
 automatizaci\'on de las tareas realizadas en una computadora, es posible que 
 las acciones de longitud unitaria tengan importancia, por lo que el
 \emph{Porcentaje de Precisi\'on} mostradas en las tablas \ref{tableRes1} y
 \ref{tableRes2}, puede variar dependiendo del caso de uso.

La lista de tareas obtenida se puede ver incrementada con el uso de estas, por 
 el hecho de que, al hacer uso de las tareas guardadas, el usuario le indica a 
 la maquina que tarea realizar, en lugar de hacerla \'el mismo, est\'e es otro 
 aspecto para destacar ya que solo se crean las secuencias, no se ejecutan 
 automaticamente, si se desea ejecutar alguna tarea guardada, el usuario es el 
 que debe ejecutarla manualmente.


Considerando el objetivo a cumplir, el software desarrollado tiene mucha 
 similitud con un creador de macros, por 
 ejemplo, Pulovers Macro Creator, las diferencias que hay que destacar se 
 mencionan en la tabla \ref{vsmacros}, en la que se realiza un an\'alisis 
 comparativo general entre ambos desarrollos. Tambi\'en, existe mucha 
 similitud con el objetivo de la metodolog\'ia de RPA, 
 por lo que en el an\'alisis 
 comparativo entre las metodolog\'ias, presentado en la tabla \ref{vsrpa}, se 
 presentan diferencias destacables entre ambas.
 

\begin{table}[h]
\centering
\begin{tabular}{m{6cm}|m{6cm}}
\hline
\textbf{Creador de macros} 	&	\textbf{Software desarrollado} \\
\hline
Hay que indicar manualmente cuando empieza y termina la acci\'on deseada	
 &	
Se monitorea cada acci\'on realizada por el usuario.\\
\hline

El usuario graba manualmente la tarea que desea automatizar	
 &
Se muestra al usuario las acciones que realiza con mayor frecuencia para que
  \'el decida cual guardar\\
\hline

El usuario requiere conocimiento del software para crear tareas complejas 	
 &
El usuario no requiere editar las tareas\\
\hline
\end{tabular}
\caption{An\'alisis comparativo del software con un generador de macros.}
\label{vsmacros}
\end{table}


\begin{table}[h]
\centering
\begin{tabular}{m{6cm}|m{6cm}}
\hline
\textbf{Robotic Process Automation}    &    \textbf{Software desarrollado} \\

\hline
Hay que indicar manualmente cuando empieza y termina la acci\'on deseada.
&
Se monitorea cada acci\'on realizada por el usuario.\\

\hline
Por medio de t\'ecnicas de reconocimiento de im\'agenes y monitoreo a los dispositivos de E/S, se determina la acci\'on realizada y el momento de ejecuci\'on. 
&
Por medio del an\'alisis en tiempo ejecuci\'on de un grafo dirigido se obtienen las tareas realizadas.\\

\hline
Se automatiza un proceso en espec\'ifico.
&
Se automatiza la tarea que m\'as realice el usuario.\\

\hline
\end{tabular}
\caption{An\'alisis comparativo de la propuesta con Robotic Process
 Automation.}
\label{vsrpa}
\end{table}


Finalmente, la metodolog\'ia desarrollada es una forma autom\'atica de obtener 
 secuencias, se ha observado que, para esta aplicaci\'on, los resultados son 
 prometedores y subjetivos, ya que depende del usuario la definici\'on de las 
 tareas que le puedan ser de utilidad, sin embargo, es posible presentar el 
 proyecto como una forma autom\'atica de realizar la automatizaci\'on de 
 procesos con RPA o de un creador de macros, por medio de aprendizaje 
 autom\'atico.