\subsection{Discusi\'{o}n de resultados}


Los resultados experimentales demuestran que el software desarrollado es capaz de proporcionar tareas \'utiles para la automatizaci\'on de las mismas, independientemente del software que este usando la persona.


Considerando que los criterios mencionados son para darle un uso espec\'ifico al grafo, es destacable que pese al entorno variable en el cual fue evaluado el software, solo fue necesario asignar unas pocas condiciones generales para obtener resultados satisfactorios. 


Entre las secuencias rechazadas se pueden localizar tareas que tienen la longitud de una acci\'on, las cuales es posible que para alg\'un usuario sean de utilidad, se presenta la siguiente lista a modo de ejemplo.
\\
\\
Secuencia 1:\\
Mouse,Scrolled,Down\\
\\
Secuencia 2:\\
Mouse,Scrolled,Up\\

Cabe recordar que la principal intenci\'on de este software es el apoyar a las personas con discapacidad en brazos y manos, teniendo en cuenta esto, las secuencias mostradas pueden ser de utilidad a alguna persona, sin embargo, considerando que la secuencia m\'as \'util es la que contiene mas elementos estas acciones unitarias fueron descartadas. Adicionalmente, si se piensa en alg\'un otro uso para el algoritmo, aparte de la automatizaci\'on de las tareas realizadas en una computadora, es posible que las acciones unitarias e impares tengan importancia, por lo que la precisi\'on mostradas en las tablas \ref{tableRes1} y \ref{tableRes2}, puede variar dependiendo del caso de uso.


Otro aspecto para destacar es que solo se crean las secuencias, si se desea ejecutar alguna tarea guardada, el usuario es el que debe ejecutarla manualmente. En el caso dado de que se desee implementar una t\'ecnica de inteligencia artificial para que la ejecuci\'on de estas tareas se realice de forma autom\'atica, es posible de hecho se plantea que este m\'etodo puede ser utilizado para obtener las reglas para el aprendizaje por demostraci\'on [Articulo].   


La lista de tareas obtenida se puede ver incrementada con el uso de estas, por el hecho de que, al hacer uso de las tareas guardadas, el usuario le indica a la maquina que tarea realizar, en lugar de hacerla \'el mismo. 


Es posible hacer mejoras al software, por ejemplo, en lugar de que se tenga que hacer clic en una ventana para realizar la tarea, seria de mayor ayuda a un discapacitado hacer uso de otro dispositivo de entrada como ser\'ia el caso de un micr\'ofono, permitiendo que se guarden y ejecuten las secuencias por medio de comandos de voz. 
