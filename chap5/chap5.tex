% ********** cap�tulo 5 **********
\chapter{Experimentos y resultados}
\label{sec:chapter5}
En el presente cap�tulo se propone el conjunto e experimentos a real�izar para demostar ...

\section{Estudio comparativo...}

%\abstract{
Los algoritmos de ...

\subsection{Introducci\'{o}n}
\label{sec:1}
Las t�cnicas de ...\\

Recientemente se consider� ... \\

En este trabajo se presentan las principales consideraciones de implantaci�nde ... 
 


\subsection{Experimentos y resultados}
\subsubsection{M\'{e}tricas para ...}
Para evaluar el desempe�o de ...
\begin{itemize}
\item Costo computacional
\item Rapidez
\item Ventaja (speedup)
\item Eficiencia 
\end{itemize}
El costo computacional, $C$, lo definimos ...
\begin{equation}
V =\frac{1}{C}
\label{eq:06}
\end{equation}
La ventaja (speedup) es la tasa que resulta de dividir la rapidez de la variante de inter�s  entre la rapidez de la variante de referencia (por ejemplo, la variante secuencial).
\begin{equation}
S =\frac{V_{obj}}{V_{ref}}
\label{eq:06}
\end{equation}
Finalmente...  


\subsection{Desarrollo experimental}
Las pruebas se realizaron en el sistema mencionado en la secci�n \ref{sim:sist}.

Se llevaron a cabo una serie de experimentos para ...


\subsection{Discusi\'{o}n de resultados}


En lo referente a...\\



En general, los experimentos demostraron que el desempe�o de ...\\


Resumiendo, de los resultados experimentales podemos destacar, en primer lugar, el buen desempe�o que ...\\


Los resultados experimentales tambi�n confirman ...


\subsection{An\'{a}lisis de resultados}
En este trabajo se present� ...

 
\subsection{Discusi\'{o}n}
Las pruebas de desempe�o indican que ...\\

% ********** End of chapter **********
