% ********** capitulo 5 **********
\chapter{Experimentos y resultados}
\label{sec:chapter5}

En el presente capitulo se describe el conjunto de experimentos realizados para evaluar la propuesta planteada ante la problem\'atica de la discapacidad que presentan las personas con problemas de movilidad en los brazos o manos.


\section{Introducci\'on}

Los algoritmos mostrados en el cap\'itulo \ref{sec:chapter1} proponen formas de automatizar actividades realizadas por una persona, ser\'ia posible implementar ese tipo de soluci\'on al problema en cuesti\'on, pero esto, representar\'ia un problema a\'un mayor, dado que se tendr\'ia que conocer las actividades que realiza el usuario del equipo para poder proponer la automatizaci\'on de \'estas. 


La propuesta planteada en este escrito propone un aprendizaje progresivo, discriminando las acciones menos frecuentes realizadas por el usuario y descartando por completo aquellas que ni se realizan. 

\subsection{Experimentos y resultados}

El uso del software no se ve limitado a las personas con movilidad reducida, este le puede servir incluso a las personas que no tengan discapacidad alguna, ya que no va por una tarea objetivo espec\'ifica, sino por las actividades que realice la persona de forma frecuente sin importar cuales sean. Con esto en mente, se logr\'o obtener 4 archivos con la lista de acciones que realizaron 4 individuos sin discapacidades en sus actividades cotidianas en una computadora.


En el periodo de aproximadamente 3 meses comprendido del 24/02/2018 al 30/05/2018 las personas tuvieron encendida su computadora diferente cantidad de tiempo el cual se muestra en la segunda columna de la tabla \ref{infodata}. Como ya se explic\'o anteriormente en el cap\'itulo \ref{sec:chapter4} una acci\'on realizada por el usuario genera un nodo, esta cantidad es mostrada en la columna 3, mientras que la columna 4 es el n\'umero m\'aximo de veces que se repiti\'o un nodo y para mantener el anonimato de las personas participantes en esta prueba, se les har\'a referencia por un n\'umero de ahora en adelante como se muestra en la primer columna. 


\begin{table}[]
\centering
\begin{tabular}{cccc}
\hline
		No. de Sujeto	&   Tiempo de Uso (Hr:Min)		&	N\'umero de Nodos	&   Repeticiones 	\\   \hline
		1				&	166:23 						&	1,494,792			&	46,036				\\
		2				&	490:24						&	1,333,016			&	116,001				\\
		3				&	1060:48						&	1,448,016			&	378,541				\\
		4				&	148:23						&	972,828				&	56,606				\\ 
\hline
\end{tabular}
\caption{Informaci\'on de los datos recabados}
\label{infodata}
\end{table}

A cada lista de acciones se le aplico el algoritmo explicado en el cap\'itulo \ref{sec:chapter4} y como se muestra en la figura \ref{fig:conc02} se utiliz\'o un factor de 70 incidencias por nodo para ser candidato para formar una secuencia y un m\'inimo de 5 repeticiones por secuencia para declarar que la tarea es \'util. 


Las secuencias de acciones que empieza con la acci\'on \emph{Release} fueron descartadas, ya que esta acci\'on implica que se qued\'o presionada una tecla o bot\'on espec\'ifico, tambi\'en se rechazaron las secuencias que terminen con la acci\'on \emph{Pressed} ya que esto dejar\'ia presionada la tecla o bot\'on hasta que se presione f\'isicamente o se llame la acci\'on \emph{Release}, las dem\'as tareas son las que se consideran como v\'alidas, como filtro adicional se limit\'o a obtener secuencias de acciones con longitudes en m\'ultiplos de 2, considerando que una tecla o bot\'on presionado debe de ser liberado para concluir la acci\'on, sin embargo, pese a esta limitante las secuencias a obtener naturalmente son; desde 1 acci\'on (tabla \ref{tableRes1}) y desde 2 acciones (tabla \ref{tableRes2}). En ambas tablas mencionadas se muestra la cantidad de secuencias aceptadas, rechazadas de acuerdo con los criterios mencionados anteriormente y el procentaje correspondiente a las secuencias aceptadas respecto al total encontrado.

Por el hecho de que este es un m\'etodo de aprendizaje acumulativo, as\'i sea que la primera secuencia sea del agrado del usuario o no, la siguiente secuencia puede contener la primera con alguna acci\'on adicional y es mas probable que esta variante ya sea de utilidad al usuario. 


\begin{table}[]
\centering
\begin{tabular}{cccc}
\hline
		No. de Sujeto	&	Secuencias Aceptadas	&   Secuencias Rechazadas	&	Asertividad		\\ \hline
		1				&	189						&	336						&	36.00 \%		\\
		2				&	165						&	322						&	33.88 \%		\\
		3				&	151						&	316						&	32.33 \%		\\
		4				&	56						&	124						&	31.11 \%		\\
\hline
\end{tabular}
\caption{Tabla de resultados con secuencias de una longitud m\'inima de 1 acci\'on}
\label{tableRes1}
\end{table}



\begin{table}[]
\centering
\begin{tabular}{cccc}
\hline
		No. de Sujeto	&	Secuencias Aceptadas	&   Secuencias Rechazadas	&	Asertividad		\\ \hline
		1				&	179						&	231						&	43.65 \%		\\
		2				&	170						&	207						&	45.09 \%		\\
		3				&	154						&	192						&	44.50 \%		\\
		4				&	52						&	67						&	43.69 \%		\\
\hline
\end{tabular}
\caption{Tabla de resultados con secuencias de una longitud m\'inima de 2 acciones}
\label{tableRes2}
\end{table}


Un ejemplo de las tareas encontradas por el software son las siguientes:
\\
\\
Sujeto: 1	\\
Descripci\'on: La tarea es utilizada en el software Blender para girar el objeto en el eje Y.	\\
Secuencia obtenida:\\
Keyboard,Pressed,G\\
Keyboard,Release,G\\
Keyboard,Pressed,Y\\
Keyboard,Release,Y\\
\\
Sujeto: 2	\\
Descripci\'on: En Windows es utilizada esta combinaci\'on de teclas para cambiar entre las ventanas abiertas.	\\
Secuencia obtenida:\\
Keyboard,Pressed,Key.alt\_l	\\
Keyboard,Pressed,Key.tab	\\
Keyboard,Release,Key.tab	\\
Keyboard,Release,Key.alt\_l	\\
\\
Sujeto: 3	\\
Descripci\'on: es la palabra ``el''.	\\
Secuencia obtenida:\\
Keyboard,Pressed,e	\\
Keyboard,Release,e	\\
Keyboard,Pressed,l	\\
Keyboard,Release,l	\\
\\
Sujeto: 4	\\
Descripci\'on: En Windows, selecciona el objeto se\~nalado por el puntero y obtiene el men\'u contextual de ese objeto.	\\
Secuencia obtenida:\\	
Mouse,Pressed,Button.left	\\
Mouse,Released,Button.left	\\
Mouse,Pressed,Button.right	\\
Mouse,Released,Button.right	\\


\subsection{Discusi\'{o}n de resultados}


Los resultados experimentales demuestran que el software desarrollado es capaz
 de proporcionar tareas \'utiles para la automatizaci\'on de las mismas,
 independientemente del software que este usando la persona, de forma que en 
 una lista pseudo--infinita de datos ordenados por momento de aparici\'on es
 posible encontrar secuencias de informaci\'on coherente para el usuario.


Considerando que los criterios mencionados son para darle un uso espec\'ifico
 al grafo, es destacable que pese al entorno variable en el cual fue evaluado
 el software, como ya se mencion\'o, solo fue necesario asignar unas pocas
 condiciones generales para obtener las secuencias coherentes.
 
 
Tambi\'en, se destaca que el sujeto \textbf{n\'umero 4} (ver tabla 
 \ref{infodata}, \ref{tableRes1} y \ref{tableRes2}) pese a ser el que menos
 tiempo de uso y secuencias reconocidas obtuvo, su porcentaje de asertividad no
 var\'ia tanto con respecto al de los dem\'as sujetos, lo cual demuestra la 
 estabilidad del algoritmo mostrado, porque, considerando los datos de la tabla 
 \ref{infodata}, el sujeto \textbf{n\'umero 4} realiz\'o mas repeticiones que el 
 sujeto \textbf{n\'umero 1} de la reducida cantidad de acciones realizadas, es 
 decir, el sujeto \textbf{n\'umero 4} realiz\'o acciones similares cada vez que 
 usaba la computadora.


Entre las secuencias rechazadas se pueden localizar tareas que tienen la
 longitud de una acci\'on, las cuales es posible que para alg\'un usuario sean
 de utilidad, se presenta la siguiente lista a modo de ejemplo.
\\
\\
Secuencia 1:\\
Mouse,Scrolled,Down\\
\\
Secuencia 2:\\
Mouse,Scrolled,Up\\

Cabe recordar que la principal intenci\'on de este software es el apoyar a las
 personas con discapacidad en brazos y manos, teniendo en cuenta esto, las
 secuencias mostradas pueden ser de utilidad a alguna persona, sin embargo,
 considerando que la secuencia m\'as \'util es la que contiene mas elementos,
 estas acciones unitarias fueron descartadas. Adicionalmente, si se piensa en
 alg\'un otro uso para el algoritmo, aparte de la automatizaci\'on de las
 tareas realizadas en una computadora, es posible que las acciones de longitud
 unitaria e impares tengan importancia, por lo que la asertividad mostradas en
 las tablas \ref{tableRes1} y \ref{tableRes2}, puede variar dependiendo del
 caso de uso.

La lista de tareas obtenida se puede ver incrementada con el uso de estas, por 
 el hecho de que, al hacer uso de las tareas guardadas, el usuario le indica a 
 la maquina que tarea realizar, en lugar de hacerla \'el mismo, est\'e es otro 
 aspecto para destacar ya que solo se crean las secuencias, no se ejecutan 
 automaticamente, si se desea ejecutar alguna tarea guardada, el usuario es el 
 que debe ejecutarla manualmente. En el caso dado de que se desee implementar 
 una t\'ecnica de inteligencia artificial para que la ejecuci\'on de estas 
 tareas se realice de forma autom\'atica, ser\'ia posible o incluso, el 
 utilizar esta metodolog\'ia para la obtenci\'on de pol\'iticas para RL o 
 aprendizaje por demostraci\'on.


El software desarrollado tiene mucha similitud con un creador de macros, por 
 ejemplo, Pulovers Macro Creator, las diferencias que hay que destacar se 
 mencionan en la tabla \ref{vsmacros}, en la que se realiza un an\'alisis 
 comparativo general entre ambos desarrollos.
 

\begin{table}[h]
\centering
\begin{tabular}{m{6cm}|m{6cm}}
\hline
Creador de macros 	&	Software desarrollado \\
\hline
Hay que indicar manualmente cuando empieza y termina la acci\'on deseada	
 &	
Se muestra al usuario las acciones que realiza con mayor frecuencia para que
  \'el decida cual guardar\\
\hline

El usuario graba manualmente la tarea que desea automatizar	
 &
La tarea mostrada no siempre es lo requerida por el usuario\\
\hline

El usuario requiere conocimiento del software para crear tareas complejas 	
 &
El usuario no requiere editar las tareas\\
\hline
\end{tabular}
\caption{An\'alisis comparativo del software con un generador de macros.}
\label{vsmacros}
\end{table}




\begin{comment}

\subsection{Discusi\'{o}n de resultados}


En lo referente a...\\



En general, los experimentos demostraron que el desempeno de ...\\


Resumiendo, de los resultados experimentales podemos destacar, en primer lugar, el buen desempeno que ...\\


Los resultados experimentales tambien confirman ...


\subsection{An\'{a}lisis de resultados}
En este trabajo se presento ...

 
\subsection{Discusi\'{o}n}
Las pruebas de desempeno indican que ...\\

% ********** End of chapter **********
\end{comment}