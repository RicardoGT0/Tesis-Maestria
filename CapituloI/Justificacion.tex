
\section{Justificación}
A nivel mundial, la discapacidad va en aumento dado que la población está
 envejeciendo y son pocos los programas privados y gubernamentales que apoyan a
 este grupo de personas\cite{OrganizacionMundialdelaSalud2011}. 
 Con referencia a los datos obtenidos del Censo de Población
 y Vivienda de 2010 era poco más del 5\% de la Población de México la que
 presentaba algún tipo de discapacidad, pero se puede apreciar que esta cifra va
 en aumento ya que en la Encuesta Nacional de Ingresos y Gasto de los Hogares
 (ENIGH) de 2012 fue el 6.6\% el porcentaje de la población la que tenía alguna
 discapacidad\cite{Milosavljevic2014}.
 

%Falta Escribir

 
La Organización Mundial de la Salud (OMS) y el Banco Mundial
 propusieron en 2011 \cite{OrganizacionMundialdelaSalud2011} una
 estrategia de colaboración entre el sector privado y gubernamental para
 rehabilitar e incorporar a la sociedad a las personas discapacitadas y así
 poder aprovechar el potencial de toda esta gente.Algunas de estas propuestas
 tienen por objetivo proporcionar accesibilidad en los servicios convencionales
 , por ejemplo transporte y educación, así como adiestrar a los servidores
 públicos, para que las personas sean tratadas con los cuidados necesarios.
 