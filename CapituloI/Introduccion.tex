
Como nos ha marcado la experiencia, la computadora al igual que cualquier otra
 máquina fue diseñada para facilitar la vida de las personas con las tareas
 repetitivas, ya sea acelerando o automatizando tareas, así mismo se ha llegado
 a un punto en la operación de la computadora en la que se realizan tareas de
 forma mecanizada ya que no hay variantes en estas.  

Los desarrolladores de software han contribuido a la automatización de estas
 tareas, sin embargo, cuando el software no es específico para una persona
 sino para un sector de la población, las necesidades llegan a ser variadas
 de un usuario a otro lo que genera un software con múltiples funciones de las
 cuales cada usuario usa un conjunto diferente esto es, por lo tanto,
 mientras más genérico se quiere hacer un software, más complicado será es su
 uso. 

Algunos de los desarrollos enfocados a la automatización de acciones humanas
 con las variantes involucradas en el mundo real principalmente son 
 aplicaciones de la robótica, sin embargo, las soluciones propuestas también
 pueden ser enfocadas a un ambiente virtual.

La propuesta presentada va enfocada al apoyo de las personas con acceso a la
 computadora, pero que debido a sus capacidades físicas no puede usar el equipo
 con la misma agilidad que una persona con todas sus facultades. Estas son las
 Personas con Movilidad Reducida(PRM), que principalmente, por cuestiones
 laborales tienen que trabajar con una computadora y que por su discapacidad se
 les dificulta el uso de la misma.

Por lo tanto, en este trabajo de investigación se propone desarrollar un sistema
 que permita el monitoreo de las acciones del usuario realizadas en una
 computadora personal (PC, Personal Computer) y obtener la secuencia de acciones
 frecuentes para su posterior reproducción.