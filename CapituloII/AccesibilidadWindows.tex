\section{Opciones de accesibilidad de Microsoft Windows}

Desde hace muchas versiones de Windows, Microsoft ha puesto empeño para que su
 sistema operativo sea posible usarlo sin importar las condiciones físicas del
 usuario \cite{DanielHubbell2016}, para lo cual colocó opciones de
 accesibilidad en el panel de control.

\begin{itemize}
	\item Lupa: Aumenta el tamaño del contenido de la pantalla para que sea más
	 fácil leerlo \cite{xatakaaccesiblilidad}.
	\item Narrador:  Esta función es una ayuda auditiva que ayuda a saber cuáles
	 son las ventanas abiertas y su contenido, por medio de una voz sintetizada
	 \cite{xatakaaccesiblilidad}.
	\item Teclado en pantalla: Como su nombre lo indica es un teclado virtual
	 con el cual podemos escribir presionando la pantalla, en caso de ser un
	 dispositivo táctil, o presionando los botones con el ratón
	 \cite{xatakaaccesiblilidad}.
	 \item Contraste alto: Otra ayuda visual para poder distinguir mejor los
	  elementos en pantalla modificando el contraste de Windows
	  \cite{xatakaaccesiblilidad}.
	  \item Reconocimiento de voz: Es una herramienta de dictado con la cual se
	   puede escribir y manipular aplicaciones e incluso el mismo sistema
	   operativo por medio de comandos de voz \cite{support14213}.
\end{itemize}
	   
Aunque no esta considerado dentro de las opciones de accesibilidad de Windows,
 el asistente Cortana facilita la realización de algunas tareas por medio de
 comandos de voz, por ejemplo, apertura de programas, la manipulación de
 recordatorios, mensajes de texto y correo electrónico \cite{support17214}. 
