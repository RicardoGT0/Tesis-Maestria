
\usepackage[utf8x]{inputenc}
\usepackage[T1]{fontenc}			%Para que detecte los acentos
\usepackage[spanish]{babel} 
%\usepackage[none]{hyphenat}		%Libreria para no cortar palabras(no Hipenacion). Usar con Sloppy
%\usepackage{resizebox}
\usepackage{amsmath}
\usepackage{amsthm}
\usepackage{graphicx}
\usepackage{color}
\usepackage{inputenc}
\usepackage{amsfonts}
\usepackage{amssymb}
\usepackage{geometry}
\usepackage{epsfig}
\usepackage{array}
\usepackage{adjustbox}
\usepackage{url}

\usepackage{colortbl} 						% Para el color en la tabla. Junto con color
\usepackage[Lenny]{fncychap}

%\usepackage[bookmarks=true]{hyperref} 		%Pone los marcadores en el pdf. Compilar con: dvi->ps, ps->pdf
\usepackage[hidelinks]{hyperref}			%Ocultar recuadros de vinculos
\usepackage{bookmark}                 		%Pone los marcadores en el pdf. Compilar con: dvi->ps, ps->pdf

%Cambiar color de vinculos
\usepackage{xcolor}

\hypersetup
{
    colorlinks,
    linkcolor={red!50!black},
    citecolor={blue!50!black},
    urlcolor={blue!80!black}
}

\usepackage{listings}	%Insertar programas en C
\usepackage{graphicx} %Utilizar figuras e imagenes
\usepackage{float}
\usepackage{subfig}		%Multiples imagenes en una figura
\usepackage{epstopdf} %Utilizar figuras EPS desde PDF
\usepackage{multirow}	%Para generar tablas
\usepackage{pifont}

\usepackage{booktabs}
\usepackage{multirow,bigstrut}
\usepackage{tabu}

\newtheorem{defin}{Definición}
\pagestyle{headings}	%Coloca los capítulos o secciones en las paginas
\geometry{tmargin=2.5cm, lmargin=2.5cm, rmargin=2.5cm, bmargin=1.5cm}   %Coloca los márgenes


% En caso de que se separe inadecuadamente la palabra usar:
\hyphenation{configu-rado}

%evitar que LaTeX distribuya los espacios en blanco a lo largo de la hoja
\raggedbottom

