
\section{Justificaci\'on}
A nivel mundial, la discapacidad va en aumento dado que la poblaci\'on est\'a
 envejeciendo y son pocos los programas privados y gubernamentales que apoyan a
 este grupo de personas\cite{OrganizacionMundialdelaSalud2011}. 
 Con referencia a los datos obtenidos del Censo de Poblaci\'on
 y Vivienda de 2010 era poco m\'as del 5\% de la Poblaci\'on de M\'exico la que
 presentaba alg\'un tipo de discapacidad, pero se puede apreciar que esta cifra
 va en aumento ya que en la Encuesta Nacional de Ingresos y Gasto de los 
 Hogares (ENIGH) de 2012 fue el 6.6\% de la poblaci\'on la que ten\'ia alguna
 discapacidad\cite{Milosavljevic2014}.
 

%Falta Escribir

 
La Organizaci\'on Mundial de la Salud (OMS) y el Banco Mundial
 propusieron en 2011 \cite{OrganizacionMundialdelaSalud2011} una
 estrategia de colaboraci\'on entre el sector privado y gubernamental para
 rehabilitar e incorporar a la sociedad a las personas discapacitadas y as\'i
 poder aprovechar el potencial de toda esta gente.Algunas de estas propuestas
 tienen por objetivo proporcionar accesibilidad en los servicios convencionales
 , por ejemplo transporte y educaci\'on, as\'i como adiestrar a los servidores
 p\'ublicos, para que las personas sean tratadas con los cuidados necesarios.
 