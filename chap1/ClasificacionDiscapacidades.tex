%\section{Discapacidades y su clasificaci\'on}

``Siguiendo el criterio del grupo de Washington, se identifica a la
 poblaci\'on con discapacidad como aquella que declara no poder hacer, 
 o tener dificultades graves para realizar actividades consideradas 
 b\'asicas''\cite{INEGI2014}


La Organizaci\'on Mundial de la Salud (OMS) con el objetivo de 
 establecer un ``lenguaje unificado y
 estandarizado''\cite{9241545429} entre otros, categoriz\'o las 
 discapacidades en el documento CIF (clasificaci\'on internacional 
 del funcionamiento, la discapacidad y la salud), en tres aspectos
 principales \cite{9241545429}:


\begin{itemize}
	\item Funciones y estructuras corporales
	\item Actividades
	\item Participaci\'on
\end{itemize}


A partir de las cuales se obtuvieron los siguientes tipos y porcentajes 
 de discapacidad en M\'exico en el a\~no 2014\cite{INEGI2014}:


\begin{itemize}
	\item {Caminar, subir o bajar usando sus piernas: $64.1\%$ }
	\item {Ver (aunque use lentes): $58.4\%$ }
  	\item {Aprender, recordar o concentrarse: $38.8\%$ } 
	\item {Escuchar (aunque use aparato auditivo): $33.5\%$ } 
	\item {Mover o usar sus brazos o manos: $33.0\%$ }
 	\item {Ba\~narse, vestirse o comer: $23.7\%$ }
	\item {Problemas emocionales o mentales: $19.6\%$ }
	\item {Hablar o comunicarse: $18.0\%$ }
\end{itemize}


``En 2014, resid\'ian en el pa\'is aproximadamente 120 millones de personas
 \ldots La prevalencia de la discapacidad en M\'exico para 2014 es de $6\%$, 
 seg\'un los datos de la ENADID 2014. Esto significa que 7.1 millones de 
 habitantes del pa\'is no pueden o tienen mucha dificultad para hacer 
 alguna de las ocho actividades evaluadas''\cite{INEGI2014}


Con los datos obtenidos en 2014 por la ENADID (Encuesta Nacional de la
 Din\'amica Demogr\'afica) se concluye que ``\ldots la poblaci\'on con 
 discapacidad muestra la estrecha relaci\'on de esta condici\'on con el 
 proceso de envejecimiento demogr\'afico''\cite{INEGI2014}, esto se puede 
 observar en las tablas \ref{discapacidadGrupoEdadP1} y 
 \ref{discapacidadGrupoEdadP2} con el porcentaje de cada grupo de personas 
 con discapacidad segun el tipo de discapacidad. 


\begin{table}[]
\centering
\scalebox{0.75}{
\begin{tabular}{lllll}
\hline
\multicolumn{1}{|c}{} 
& \multicolumn{4}{|c|}{\textbf{Tipo de discapacidad}}
\\ \cline{2-5} 

\multicolumn{1}{|m{3cm}}{\textbf{Grupo de edad}}
& \multicolumn{1}{|m{3cm}|}{\textbf{Caminar, subir o bajar usando sus
 piernas}} 
& \multicolumn{1}{m{3cm}|}{\textbf{Ver (aunque use lentes)}} 
& \multicolumn{1}{m{3cm}|}{\textbf{Mover o usar sus brazos o manos}} 
& \multicolumn{1}{m{3cm}|}{\textbf{Aprender, recordar o concentrarse}} 
\\ \hline

\multicolumn{1}{|m{3cm}|}{\textbf{Ni\~nos \hspace{2cm}(0 a 14 a\~nos)}}
& \multicolumn{1}{r|}{$36.2\%$}         
& \multicolumn{1}{r|}{$26.9\%$}         
& \multicolumn{1}{r|}{$14.1\%$}         
& \multicolumn{1}{r|}{$40.8\%$}
\\ \hline

\multicolumn{1}{|m{3cm}|}{\textbf{J\'ovenes \hspace{2cm}(15 a 29 a\~nos)}}
& \multicolumn{1}{r|}{$32.1\%$}         
& \multicolumn{1}{r|}{$44.6\%$}         
& \multicolumn{1}{r|}{$18.2\%$}
& \multicolumn{1}{r|}{$31.5\%$}
\\ \hline

\multicolumn{1}{|m{3cm}|}{\textbf{Adultos \hspace{2cm}(30 a 59 a\~nos)}}
& \multicolumn{1}{r|}{$56.2\%$}
& \multicolumn{1}{r|}{$58.2\%$}
& \multicolumn{1}{r|}{$28.5\%$}
& \multicolumn{1}{r|}{$32.1\%$}
\\ \hline

\multicolumn{1}{|m{3cm}|}{\textbf{Adultos mayores (60 a\~nos y mas)}}
& \multicolumn{1}{r|}{$81.3\%$}
& \multicolumn{1}{r|}{$67.2\%$}
& \multicolumn{1}{r|}{$42.7\%$}
& \multicolumn{1}{r|}{$44.6\%$}        
\\ \hline

\multicolumn{1}{|m{3cm}|}{\textbf{Total}}                        
& \multicolumn{1}{r|}{$64.1\%$}        
& \multicolumn{1}{r|}{$58.4\%$}        
& \multicolumn{1}{r|}{$33.0\%$}         
& \multicolumn{1}{r|}{$38.8\%$}         
\\ \hline
\end{tabular}
}
\caption{Porcentaje de la poblaci\'on con discapacidad, por grupo de edad
 seg\'un tipo de discapacidad 1ra parte\cite{INEGI2014}.}
\label{discapacidadGrupoEdadP1}
\end{table}


\begin{table}[]
\centering
\scalebox{0.75}{
\begin{tabular}{lllll}
\hline
\multicolumn{1}{|c}{} 
& \multicolumn{4}{|c|}{\textbf{Tipo de discapacidad}}
\\ \cline{2-5} 

\multicolumn{1}{|m{3cm}}{\textbf{Grupo de edad}}             
& \multicolumn{1}{|m{3cm}|}{\textbf{Escuchar(aunque use aparato auditivo)}} 
& \multicolumn{1}{m{3cm}|}{\textbf{Ba\~narse, vestirse o comer}} 
& \multicolumn{1}{m{3cm}|}{\textbf{Hablar o comunicarse}} 
& \multicolumn{1}{m{3cm}|}{\textbf{Problemas emocionales o mentales}} 
\\ \hline

\multicolumn{1}{|m{3cm}|}{\textbf{Ni\~nos \hspace{2cm}(0 a 14 a\~nos)}}
& \multicolumn{1}{r|}{$13.4\%$}         
& \multicolumn{1}{r|}{$37.4\%$}         
& \multicolumn{1}{r|}{$45.6\%$}         
& \multicolumn{1}{r|}{$26.6\%$}         
\\ \hline

\multicolumn{1}{|m{3cm}|}{\textbf{J\'ovenes \hspace{2cm}(15 a 29 a\~nos)}}
& \multicolumn{1}{r|}{$18.5\%$}         
& \multicolumn{1}{r|}{$16.4\%$}         
& \multicolumn{1}{r|}{$28.5\%$}        
& \multicolumn{1}{r|}{$28.0\%$}       
\\ \hline

\multicolumn{1}{|m{3cm}|}{\textbf{Adultos \hspace{2cm}(30 a 59 a\~nos)}} 
& \multicolumn{1}{r|}{$24.2\%$}         
& \multicolumn{1}{r|}{$14.5\%$}         
& \multicolumn{1}{r|}{$13.4\%$}         
& \multicolumn{1}{r|}{$20.1\%$}         
\\ \hline

\multicolumn{1}{|m{3cm}|}{\textbf{Adultos mayores (60 a\~nos y mas)}}
& \multicolumn{1}{r|}{$46.9\%$}         
& \multicolumn{1}{r|}{$29.3\%$}         
& \multicolumn{1}{r|}{$14.0\%$}         
& \multicolumn{1}{r|}{$16.3\%$}         
\\ \hline

\multicolumn{1}{|m{3cm}|}{\textbf{Total}}  
& \multicolumn{1}{r|}{$33.5\%$}         
& \multicolumn{1}{r|}{$23.7\%$}         
& \multicolumn{1}{r|}{$18.0\%$}         
& \multicolumn{1}{r|}{$19.6\%$}         
\\ \hline
\end{tabular}
}
\caption{Porcentaje de la poblaci\'on con discapacidad, por grupo de edad
 seg\'un tipo de discapacidad 2da parte\cite{INEGI2014}.}
\label{discapacidadGrupoEdadP2}
\end{table}


Entre causales cuantificadas se encuentran las siguientes con sus respectivos
 porcentajes:

\begin{itemize}
	\item Enfermedad: $41.3\%$
	\item Edad avanzada: $33.1\%$
	\item Nacimiento: $10.7\%$
	\item Accidente: $8.8\%$
	\item Violencia: $0.6\%$
	\item Otra causa: $5.5\%$
\end{itemize}


El apartado de \emph{Otra causa} es para cuando la causa es por factores 
 ambientales y contextuales. En la tabla \ref{discapacidadCausa} se puede 
 apreciar cual es la causa mas probable de alguna discapacidad en 
 espec\'ifico, por ejemplo, para \emph{escuchar (aunque use aparato 
 auditivo)} la causa mas probable es por \emph{edad avanzada}. 


\begin{table}[]
\centering
\scalebox{0.55}{
\begin{tabular}{lllllll}
\hline
\multicolumn{1}{|c}{} 
& \multicolumn{6}{|c|}{\textbf{Causa de la discapacidad}}\\ \cline{2-7}
 
\multicolumn{1}{|m{3cm}}{\textbf{Tipo de discapacidad}}
& \multicolumn{1}{|m{3cm}|}{\textbf{Enfermedad}}
& \multicolumn{1}{m{3cm}|}{\textbf{Edad Avanzada}}
& \multicolumn{1}{m{3cm}|}{\textbf{Nacimiento}} 
& \multicolumn{1}{m{3cm}|}{\textbf{Accidente}} 
& \multicolumn{1}{m{3cm}|}{\textbf{Violencia}} 
& \multicolumn{1}{m{3cm}|}{\textbf{Otra causa}} \\ \hline

\multicolumn{1}{|m{3cm}|}{\textbf{Caminar, subir o bajar usando sus piernas}}
& \multicolumn{1}{r|}{$49.0\%$}
& \multicolumn{1}{r|}{$25.1\%$}
& \multicolumn{1}{r|}{$5.8\%$}
& \multicolumn{1}{r|}{$16.2\%$}
& \multicolumn{1}{r|}{$0.3\%$}
& \multicolumn{1}{r|}{$3.0\%$} \\ \hline

\multicolumn{1}{|m{3cm}|}{\textbf{Ver (aunque use lentes)}}
& \multicolumn{1}{r|}{$44.3\%$}
& \multicolumn{1}{r|}{$36.7\%$}
& \multicolumn{1}{r|}{$9.1\%$}
& \multicolumn{1}{r|}{$5.6\%$}
& \multicolumn{1}{r|}{$0.2\%$}
& \multicolumn{1}{r|}{$4.1\%$} \\ \hline

\multicolumn{1}{|m{3cm}|}{\textbf{Mover o usar sus brazos o manos}}
& \multicolumn{1}{r|}{$47.8\%$}
& \multicolumn{1}{r|}{$29.2\%$}
& \multicolumn{1}{r|}{$6.1\%$}
& \multicolumn{1}{r|}{$14.1\%$}
& \multicolumn{1}{r|}{$0.5\%$}
& \multicolumn{1}{r|}{$2.3\%$} \\ \hline

\multicolumn{1}{|m{3cm}|}{\textbf{Aprender, recordar o concentrarse}}
& \multicolumn{1}{r|}{$27.5\%$}
& \multicolumn{1}{r|}{$48.7\%$}
& \multicolumn{1}{r|}{$13.2\%$}
& \multicolumn{1}{r|}{$3.3\%$}
& \multicolumn{1}{r|}{$1.0\%$}
& \multicolumn{1}{r|}{$6.3\%$} \\ \hline

\multicolumn{1}{|m{3cm}|}{\textbf{Escuchar (aunque use aparato auditivo)}}
& \multicolumn{1}{r|}{$28.9\%$}
& \multicolumn{1}{r|}{$49.6\%$}
& \multicolumn{1}{r|}{$9.3\%$}
& \multicolumn{1}{r|}{$6.3\%$}
& \multicolumn{1}{r|}{$0.8\%$}
& \multicolumn{1}{r|}{$5.1\%$} \\ \hline

\multicolumn{1}{|m{3cm}|}{\textbf{Ba\~narse, vestirse o comer}}
& \multicolumn{1}{r|}{$45.6\%$}
& \multicolumn{1}{r|}{$25.9\%$}
& \multicolumn{1}{r|}{$10.1\%$}
& \multicolumn{1}{r|}{$9.5\%$}
& \multicolumn{1}{r|}{$0.4\%$}
& \multicolumn{1}{r|}{$8.5\%$} \\ \hline

\multicolumn{1}{|m{3cm}|}{\textbf{Hablar o comunicarse}}
& \multicolumn{1}{r|}{$34.6\%$}
& \multicolumn{1}{r|}{$19.9\%$}
& \multicolumn{1}{r|}{$31.8\%$}
& \multicolumn{1}{r|}{$3.6\%$}
& \multicolumn{1}{r|}{$0.6\%$}
& \multicolumn{1}{r|}{$9.5\%$} \\ \hline

\multicolumn{1}{|m{3cm}|}{\textbf{Problemas emocionales o mentales}}
& \multicolumn{1}{r|}{$45.5\%$}
& \multicolumn{1}{r|}{$16.9\%$}
& \multicolumn{1}{r|}{$18.1\%$}
& \multicolumn{1}{r|}{$4.2\%$}
& \multicolumn{1}{r|}{$2.4\%$}
& \multicolumn{1}{r|}{$12.9\%$} \\ \hline

\multicolumn{1}{|m{3cm}|}{\textbf{Total}}
& \multicolumn{1}{r|}{$41.3\%$}
& \multicolumn{1}{r|}{$33.1\%$}
& \multicolumn{1}{r|}{$10.7\%$}
& \multicolumn{1}{r|}{$8.8\%$}
& \multicolumn{1}{r|}{$0.6\%$}
& \multicolumn{1}{r|}{$5.5\%$} \\ \hline
\end{tabular}
}
\caption{Distribuci\'on porcentual de las discapacidades, por tipo seg\'un
 causa de la discapacidad\cite{INEGI2014}.}
\label{discapacidadCausa}
\end{table}
