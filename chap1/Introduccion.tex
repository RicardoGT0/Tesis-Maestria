
Como nos ha marcado la experiencia, la computadora al igual que cualquier
 otra m\'aquina fue dise\~nada para facilitar la vida de las personas con las 
 tareas repetitivas, ya sea acelerando o automatizando tareas, as\'i mismo se 
 ha llegado a un punto en la operaci\'on de la computadora en la que se 
 realizan tareas de forma mecanizada ya que no hay variantes en estas.  

Los desarrolladores de software han contribuido a la automatizaci\'on de 
 estas tareas, sin embargo, cuando el software no es espec\'ifico para una 
 persona sino para un sector de la poblaci\'on, las necesidades llegan a ser 
 variadas de un usuario a otro lo que genera un software con m\'ultiples 
 funciones de las cuales cada usuario usa un conjunto diferente, por lo 
 tanto, mientras m\'as gen\'erico se quiere hacer un software, m\'as 
 complicado ser\'a su uso. 

Algunos de los desarrollos enfocados a la automatizaci\'on de acciones 
 humanas con las variantes involucradas en el mundo real principalmente son  
 aplicaciones de la rob\'otica, sin embargo, las soluciones propuestas 
 tambi\'en pueden ser enfocadas a un ambiente virtual.

La propuesta presentada va enfocada al apoyo de las personas con acceso a la
 computadora, pero que debido a sus capacidades f\'isicas no puede usar el 
 equipo con la misma agilidad que una persona con todas sus facultades. Estas 
 son las Personas con Movilidad Reducida(PMR), que principalmente, por 
 cuestiones laborales tienen que trabajar con una computadora y que por su 
 discapacidad se les dificulta el uso de la misma.

Por lo tanto, en este trabajo de investigaci\'on se propone desarrollar un 
 sistema que permita el monitoreo de las acciones del usuario realizadas en 
 una computadora personal (PC, Personal Computer) y obtener la secuencia de 
 acciones frecuentes para su posterior reproducci\'on.
 
 