% ********** Resumen **********
\chapter{Abstract}
\label{sec:Abstract}

In the present thesis an unsupervised learning methodology is presented that 
 allows to identify repetitive sequences of a list of nominal data in which 
 the elements are in order of appearance. This tool was used to identify the 
 tasks performed by a person with reduced mobility in their arms and hands 
 on a computer and automate these tasks as much as possible so that the 
 person can increase their performance. The software was designed as a basic 
 graphic interface, however, it is sufficient, since the intention is to 
 provide an easy-to-use system and reduce the number of steps necessary to 
 perform a task on the computer equipment.


The aforementioned methodology makes use of a directed graph which is 
 constructed while the user makes use of the computer and at the moment the 
 sequences found are shown, so that he can identify if it is useful or not. 
 The actions defined as useful, are registered for their later use, allowing 
 the user to make use of them, improving their performance and the task 
 suggestions that are shown to him.


Considering the objective of the methodology in this work, it is classified    
 as a variant of Robotic Process Automation (RPA).