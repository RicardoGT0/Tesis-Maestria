\section{Trabajos futuros}


Se pueden desarrollar trabajos posteriores basados en el presente desarrollo
 mejorando la interacci\'on del usuario para que en lugar de tener que hacer 
 clic en una ventana para realizar la tarea, se realice por medio de otro 
 dispositivo de entrada por ejemplo, un micr\'ofono, lo que requiere la 
 implementaci\'on de un software de conversion de voz a texto, permitiendo que 
 se guarden y ejecuten las secuencias por medio de comandos de voz. 
 

Se propone como mejora implementar un filtro din\'amico adicional que tome en 
 consideraci\'on las acciones elegidas por el usuario y as\'i aumentar el 
 porcentaje de asertividad, lo cual implica reducir las secuencias poco
  \'utiles para \'el, otra alternativa, es implementar un sistema de
 reconocimiento de i\'agenes como se hace en RPA. 

 
En el caso dado de que se desee implementar una t\'ecnica de inteligencia  
 artificial para que la ejecuci\'on de estas tareas se realice de forma 
 autom\'atica, ser\'ia posible, o incluso, en otra vertiente, utilizar esta 
 metodolog\'ia para la obtenci\'on de pol\'iticas para RL o aprendizaje por 
 demostraci\'on.


El software identifica secuencias que son repetitivas, por lo que se 
 plantean las siguientes interrogantes ?`Qu\'e pasar\'ia si se le 
 proporciona el software a una persona con Parkinson?  
 ?`Es posible identificar un patr\'on entre sus movimientos? 
 ?`Es posible proporcionarle asistencia con este software?.
