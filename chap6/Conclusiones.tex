\section{Conclusiones}
La propuesta aqu\'{i} presentada demuestra que el algoritmo no tiene
 un tiempo de finalizaci\'on determinado, el aprendizaje es continuo, durante 
 la ejecuci\'on obtiene las secuencias y no requiere datos de ejemplo.
 
Tambi\'en, se destaca que el sujeto 4 (ver tabla \ref{infodata},
 \ref{tableRes1} y \ref{tableRes2})pese a ser el que menos tiempo de uso y 
 secuencias reconocidas obtuvo, su porcentaje de asertividad no var\'ia tanto
 con respecto al de los dem\'as sujetos, lo cual demuestra la estabilidad del 
 algoritmo mostrado. 
 
Los objetivos se cumplieron al $100\%$ y las metas planteadas al $85.72\%$. A 
 continuaci\'on se justifican dichas cifras:

\begin{itemize}
\item {
Se implement\'o un sistema de captura para el teclado y rat\'on con el cual 
 obtuv\'o la informaci\'on de 4 sujetos en el plazo de 3 meses para 
 realizar las pruebas mencionandas.
}
	
\item {
La propuesta de la implementaci\'on de una estructura arborescente fue cambiada
 a un grafo dirigido, ya que al dise\~nar el sistema se observ\'o que no se 
 cumpl\'ia con la caracter\'istica de no tener circuitos y esto fue necesario
 para ahorrar espacio en memoria y cumplir con el objetivo principal.
}

\item {
Se plante\'o la b\'usqueda de sistemas similares en la cual no se logr\'o 
 el \'exito esperado, por que no se encontraron los elementos necesarios 
 para realizar una comparaci\'on.
}
\end{itemize}

