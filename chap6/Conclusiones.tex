\section{Conclusiones}

Se implement\'o un sistema de captura para el teclado y rat\'on, con el cual
 se obtuvo la informaci\'on de 7 sujetos para realizar las pruebas mencionadas.


Se dise\~n\'o un algoritmo de aprendizaje no supervisado que no tiene un
 tiempo de finalizaci\'on determinado, el aprendizaje es continuo, durante la 
 ejecuci\'on obtiene secuencias de acciones realizadas por un usuario, sin 
 datos de ejemplo. Los resultados experimentales demuestran que el software 
 desarrollado es capaz de proporcionar tareas \'utiles para la 
 automatizaci\'on de las mismas, independientemente del software que este 
 usando la persona, de forma que en una lista pseudo--infinita de datos 
 ordenados por momento de aparici\'on, es posible encontrar secuencias de 
 informaci\'on coherente para un usuario.

	
La propuesta de la implementaci\'on de una estructura arborescente fue
 cambiada a un grafo dirigido, ya que al dise\~nar el sistema se observ\'o que 
 no se cumple con la caracter\'istica de no tener circuitos y esto fue 
 necesario para ahorrar espacio en memoria y cumplir con el objetivo 
 principal.


Se plante\'o la b\'usqueda de sistemas similares en la cual no se logr\'o 
 el \'exito esperado, dejando como sistemas similares a los creadores de
 macros y RPA, por lo que no se encontraron los elementos necesarios 
 para realizar un an\'alisis comparativo de resultados.


En la hip\'otesis propuesta se menciona que los humanos son seres de 
 costumbres por lo que el software propuesto debe de reconocer esos h\'abitos, 
 pero considerando que en las secuencias obtenidas solo hay registro de teclas
 y botones del teclado y rat\'on respectivamente, lo cual implica que los
 movimientos con el rat\'on no son tan mec\'anicos como se esperaba.
 