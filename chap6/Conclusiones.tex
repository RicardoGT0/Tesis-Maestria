\section{Conclusiones}
La propuesta aqu\'{i} presentada demuestra que el algoritmo no tiene
 un tiempo de finalizaci\'on determinado, el aprendizaje es continuo, durante 
 la ejecuci\'on obtiene las secuencias y no requiere datos de ejemplo.
 
Tambi\'en, se destaca que el sujeto 4 pese a ser el que menos tiempo de uso y 
 secuencias reconocidas obtuvo, su porcentaje de asertividad no varia tanto con 
 respecto al de los dem\'as sujetos, lo cual demuestra la estabilidad del 
 algoritmo mostrado. 
 
Los objetivos y metas planteados al inicio de este trabajo fueron cubiertos en
 su totalidad:

\begin{itemize}
\item {
Se desarroll\'o un sistema de captura para el teclado y rat\'on con el cual fue 
 posible obtener los datos de los 4 sujetos por 3 meses para las pruebas 
 mostradas.
}
	
\item {
La propuesta de la implementaci\'on de una estructura arborescente fue cambiada
 a un grafo dirigido, ya que al dise\~nar el sistema se observo que no se 
 cumpl\'ia con la caracter\'istica de no tener circuitos y esto fue necesario
 para ahorrar espacio en memoria y hacerlo m\'as eficiente, puesto que era
 menos probable que se llegara a repetir una rama completa. 
}

\item {
Se plantea la b\'usqueda de sistemas similares en la cual no se tuvo \'exito
 pese a la variedad de t\'erminos utilizados, los trabajos y metodolog\'ias
 presentadas son lo que m\'aa se asemeja, sin embargo, por la naturaleza del 
 problema no es posible implementar alguna de \'estas ya que estos buscan una
 tarea objetivo y por lo mismo tampoco fue posible realizar una comparativa de
 resultados.
}
\end{itemize}

