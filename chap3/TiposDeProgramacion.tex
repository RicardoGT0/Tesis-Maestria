\section{Tipos de programaci\'on}

Para programaci\'on orientada a objetos (POO) se tiene la siguiente
 definici\'on:


``La programaci\'on orientada a objetos es un m\'etodo de implementaci\'on
 en el que los programas se organizan como colecciones de objetos cooperativos,
 donde cada uno representa una instancia de alguna clase y \'estas clases 
 son todas miembros de una jerarqu\'ia de clases unidas por herencia'' 
 \cite{020189551X}.


Algunas de las ventajas de la POO sobre la programaci\'on procedimental
 son la modularidad y la reutilizaci\'on de c\'odigo, facilitando el 
 desarrollo de programas, como si solo se unieran ``bloques de construcci\'on'' 
 \cite{020189551X}, tambi\'en, existe el encapsulamiento de los datos, lo 
 cual permite controlar el acceso a la informaci\'on almacenada 
 \cite{8448132467}. Adem\'as, el tipo de abstracci\'on es diferente en 
 un lenguaje procedimental a uno que es orientado a objetos, 
 ya que se trata de modelar objetos del mundo real en el programa y en 
 lugar de empezar por la definici\'on de las estructuras de datos y 
 procedimientos a usar, se empieza el desarrollo pensando en el objeto y 
 cuales son las tareas a desempe\~nar de ese objeto, lo que facilita la 
 implementaci\'on de una idea \cite{8448132467}. 


``En un lenguaje orientado a objetos verdadero, toda entidad en el dominio
 del problema se expresa a trav\'es del concepto de objetos'' 
 \cite{8448132467}, entre los lenguajes orientados a objetos Python
 \cite{MarzalVar2014} y \texttt{C\#} \cite{8448132467}, mientras que
 \texttt{C++} \cite{8448132467} no es un lenguaje orientados a objetos y 
 ``\ldots java, tan bueno como es, tiene algunas limitantes como lenguaje 
 orientado a objetos'' \cite{8448132467}.


``Se dice que un sistema de software est\'a basado en eventos si sus partes
 interact\'uan principalmente usando notificaciones de eventos'' 
 \cite{9781430201564}. Los sistemas Operativos con interfaz gr\'afica est\'an 
 basados en eventos, ya que la interfaz grafica de usuario (GUI, por sus siglas en 
 ingl\'es) espera de forma pasiva a que el usuario realice alguna acci\'on sobre 
 los controles \cite{9781430201564} (botones, cuadros de texto, etc). La 
 programaci\'on basada en eventos empez\'o a tener su relevancia a principios de 
 los a\~nos 90 con el surgimiento de Microsoft Visual Basic \cite{9781430201564}.
 
  
Considerando que el paso de mensajes entre las diferentes partes de un sistema
 para poder sincronizar \'estas \cite{9781430201564} es una caracter\'istica que 
 distingue a este paradigma de la programaci\'on, aquel lenguaje de programaci\'on 
 que permita el uso de hilos y sus m\'etodos de sincronizaci\'on, se puede decir 
 que permite la programaci\'on basada en eventos, esto sin mencionar los que son 
 compatibles con alg\'un framework para desarrollar GUI similares y/o compatibles
 con Microsoft Windows.

