\subsection{RPA}

En la Automatizaci\'on de Procesos Rob\'oticos,
 se tiene por objetivo automatizar las acciones
 repetitivas en una computadora usando robots de software. RPA es una 
 metodolog\'ia en la que, por medio de im\'agenes o video, tomadas por un 
 dispositivo externo o en la misma computadora mientras se lleva a cabo el 
 proceso a automatizar, se analizan los cambios ocurridos entre una imagen y 
 otra utilizando t\'ecnicas de reconocimiento de im\'agenes, adicionalmente,
 se tiene en cuenta la acci\'on realizada por el usuario con los dispositivos
 de entrada (teclado, rat\'on o entrada t\'actil) y el tiempo que se tarda en
 sufrir un cambio entre las im\'agenes. Para la ejecuci\'on de las tareas se 
 verifica por medios visuales que el software en la computadora se encuentre 
 en el estado correcto para la ejecuci\'on el proceso deseado, por ejemplo, 
 la presencia de una ventana espec\'ifica. 

De acuerdo con la clasificaci\'on realizada por Francesco Corea, RPA est\'a 
 definido como  una ``tecnolog\'ia que extrae la lista de reglas y acciones a 
 realizar observando al usuario que realiza una determinada tarea'' 
 \cite{Corea2019}  por tal esta categorizado como una herramienta que se basa 
 en el conocimiento y la l\'ogica.
 
La implementaci\'on de RPA en una empresa proporciona la agilizaci\'on de los 
 procesos, el aumento en la precisi\'on, por tal, la reducci\'on de errores y 
 reducci\'on de costos a corto plazo, sin embargo, se requiere de mucho 
 an\'alisis y conocimiento del proceso a automatizar, por lo que tambi\'en 
 se han propuesto metodolog\'ias para an\'alisis del proceso\cite{8592629,
 10.1007/978-3-319-74030-0_51}, herramientas de procesamiento del lenguaje, 
 que a partir de una descripci\'on textual del proceso pueden indicar las 
 partes de este y quien las desarrolla\cite{10.1007/978-3-319-91704-7_5} e 
 incluso aplicar t\'ecnicas de inteligencia artificial y aprendizaje 
 m\'aquina para que la implementaci\'on de RPA sea m\'as f\'acil, 
 principalmente para personas sin conocimiento profundo del software
 \cite{Mohanty2018}.
