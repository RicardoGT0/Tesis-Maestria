\section{Aprendizaje M\'aquina}

Como se menciona en \cite{9780471056690} ``En el sentido m\'as amplio,
 cualquier m\'etodo que incorpore informaci\'on de ejemplo para el 
 entrenamiento en el dise\~no de un clasificador emplea aprendizaje''.
 Lo cual es evidente en el aprendizaje supervisado ya que en este existe 
 un maestro expl\'icito que proporciona la informaci\'on etiquetada
 \cite{9780471056690}, para que sirva de ejemplo al algoritmo de los casos 
 desconocidos que se le presentaran m\'as adelante. Pero en el aprendizaje 
 no supervisado, no requiere un maestro expl\'icito \cite{9780471056690}, sin 
 embargo, se le proporciona informaci\'on sobre la tarea, por ejemplo, el 
 n\'umero de grupos en los que hay que separar la informaci\'on proporcionada.


En los textos \cite{9780471056690, 9780387310732} se menciona el aprendizaje 
 por refuerzo (RL, por sus siglas en ingl\'es) como otra divisi\'on, ya que 
 a est\'e se le proporciona un conjunto de pol\'iticas de recompensa para 
 cumplir con un objetivo espec\'ifico, en lugar de ejemplos etiquetados como 
 en el caso del aprendizaje supervisado, estas reglas indicaran de forma 
 binaria cuales acciones le permite maximizar la recompensa por prueba y error.


El aprendizaje por demostraci\'on (LfD, por sus siglas en ingl\'es)
 \cite{ARGALL2009469} es una metodolog\'ia de aprendizaje m\'aquina que a 
 partir del ejemplo de la tarea objetivo (proporcionado por un maestro) 
 se generan las pol\'iticas necesarias, para que por medio del aprendizaje 
 por refuerzo, el aprendiz realice la tarea.


Una de las principales caracter\'isticas de esta \'ultima metodolog\'ia
 mencionada, es la forma de adquisici\'on de datos; tele-operaci\'on o 
 sombreado. Para la tele-operaci\'on\cite{ARGALL2009469}, el aprendiz va 
 a grabar desde sus propios sensores mientras que el maestro va a manipular 
 al aprendiz desde un controlador o con sus propias manos, esto proporciona 
 un mapeo directo entre la acci\'on grabada y la acci\'on a realizar. Por 
 otro lado, el sombreado\cite{ARGALL2009469} requiere un algoritmo adicional 
 de mapeo, ya que est\'e seguir\'a las acciones realizadas por el maestro por 
 medios visuales y marcadores en el maestro, mientras graba de sus propios 
 sensores. 


La forma de procesar los datos recabados en la adquisici\'on de datos para 
 obtener las pol\'iticas de recompensa para el RL puede ser por: funci\'on 
 de mapeo, plan o modelo del sistema. El usar una funci\'on de mapeo 
 \cite{ARGALL2009469} puede ser en un enfoque continuo (Regresi\'on) o 
 discreto (Clasificadores); los algoritmos de regresi\'on pueden ser a modo 
 de ejemplo, aprendizaje lento (lazy learning, en ingl\'es) o regresi\'on 
 ponderada localmente (locally weighted regression, en ingl\'es); los 
 clasificadores usados para para el enfoque discreto pueden ser KNN 
 (k-Nearest Neighbors, en ingl\'es), \'arboles de decisi\'on, redes
 neuronales, etc.
 

Para obtener las reglas a trav\'es de un modelo de sistema\cite{ARGALL2009469}
 se realiza un mapeo del mundo en el que se va a desenvolver el aut\'omata 
 y a partir de este con los datos recabados se generan las pol\'iticas de 
 recompensa. Mientras que para usar un plan\cite{ARGALL2009469}, solo se 
 considera la secuencia de acciones desde un estado inicial a un estado final 
 objetivo con estados condicionales previos y resultantes. 
