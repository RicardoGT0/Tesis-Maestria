\section{Tipos de datos}

En miner\'ia de datos, existen 2 clasificaciones para los tipos de datos
 \cite{Yang2010}, los cualitativos(por ejemplo: tipo de sangre, opini\'on 
 personal, c\'odigo postal, tel\'efono) y num\'ericos. A continuaci\'on se 
 presentan sus caracter\'isticas.

\begin{itemize}
\item Cualitativos\cite{Yang2010} 
	\begin{itemize}
	\item Son los que se pueden ubicar en distintas  categor\'ias.
	\item Algunos se pueden colocar en orden significativo
	\item No se les pueden aplicar operaciones matem\'aticas
	\end{itemize}
	 

\item Cuantitativos\cite{Yang2010}
	\begin{itemize}
	\item Son de naturaleza num\'erica. 
	\item Se pueden clasificar en orden. 
	\item Admiten operaciones aritm\'eticas significativas. 
	\item Pueden ser discretos o continuos.		
	\end{itemize}
\end{itemize}

Los datos tambi\'en se pueden clasificar por la forma en que se categorizan, cuentan o miden. Este tipo de clasificaci\'on utiliza escalas de medici\'on, y cuatro niveles comunes de escalas son: 

\begin{itemize}
\item Nominal\cite{Yang2010}:
	\begin{itemize}
	\item Se clasifican los datos en categor\'ias mutuamente excluyentes (no superpuestas) y exhaustivas en las que no se puede imponer un orden o clasificaci\'on significativa en los datos. 
	\end{itemize}
		
\item Ordinal\cite{Yang2010}:
	\begin{itemize}
	\item Se clasifican los datos en categor\'ias que se pueden clasificar. 
	\item Las diferencias entre los rangos no pueden calcularse mediante aritm\'etica.
	\item Se pueden ordenar
	\end{itemize}		

\item Intervalo\cite{Yang2010}: 
	\begin{itemize}
	\item Se clasifican los datos y las diferencias entre las unidades de medida se pueden calcular mediante aritm\'etica. 
	\item El cero en el nivel de intervalo de medici\'on no significa \emph{null} o \emph{nada} como cero en aritm\'etica.
	\end{itemize}		

\item Relaci\'on\cite{Yang2010}:
	\begin{itemize}
	\item Se clasifican los datos y las diferencias entre las unidades de medida se pueden calcular mediante aritm\'etica. 
	\item El cero en el nivel de intervalo de medici\'on significa \emph{null} o \emph{nada} como cero en aritm\'etica.
	\end{itemize}
		
\end{itemize}