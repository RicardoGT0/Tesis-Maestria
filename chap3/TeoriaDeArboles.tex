\section{Teor\'ia de \'arboles}
En t\'erminos matem\'aticos un \'arbol es un grafo $T$ el cual se puede definir de
 la siguiente forma\cite{SUSANNAS.EPP2012}:

\begin{itemize}
	\item ``Un grafo se llama \'arbol si y solo si, est\'a libre de circuitos y
	 es conexo.''
	\item ``Un v\'ertice de grado 1 en $T$ se denomina un v\'ertice terminal (o
	 una hoja).''
	\item ``Un v\'ertice de grado superior a 1 en $T$ es un v\'ertice interno (o
	 un v\'ertice de rama).''
\end{itemize}

Los arboles son muy usados en la actualidad ya que permiten  darle sentido a
 la informaci\'on contenida, gracias a la asociatividad, parentizaci\'on y
 prioridad que este permite de manera impl\'icita. Entre sus usos m\'ultiples en
 el \'area de la inform\'atica se pueden destacar los  siguientes; relaciones
 entre m\'odulos de programaci\'on, arboles de decisi\'on en inteligencia artificial
 y representaciones gram\'aticas\cite{gutierrez1999estructuras}.  

Una forma de ver a un \'arbol puede ser como un solo nodo, est\'e recibe el nombre
 de  nodo ra\'iz, al cual se le puede enraizar un sinf\'in de arboles lo que da
 origen  a un \'arbol con otras caracter\'isticas, dependiendo del resultado se le
 puede clasificar en alguno de los modelos existentes, por ejemplo; \'arbol 
 general o \'arbol binario\cite{gutierrez1999estructuras}. 

El \'arbol general es un modelo con una cantidad indeterminada de nodos hijos,
 mientras que el \'arbol binario es un caso particular del \'arbol general ya que
 este tiene la caracter\'istica de tener siempre en cada nodo 2 nodos hijos como
 m\'aximo, cabe mencionar que este es uno de los modelos a mas usados, as\'i como
 el ultimo caso mencionado, hay arboles que tienen una cantidad fija de nodos
 hijos, de manera general estos son llamados arboles n-arios
 \cite{gutierrez1999estructuras}.