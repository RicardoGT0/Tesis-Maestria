% ********** Resumen **********
\chapter{Resumen}
\label{sec:Resumen}

En la presente tesis se presenta una metodolog\'ia de aprendizaje no
 supervisado que permite identificar secuencias repetitivas de una lista de
 datos nominales en la que los elementos est\'an por orden aparici\'on. Esta 
 herramienta fue utilizada para identificar las tareas que realiza una 
 persona con movilidad reducida en brazos y manos en una computadora y 
 automatizar dichas tareas lo m\'as posible para que la persona pueda 
 aumentar su desempe\~no. El software fue dise\~nado como una interfaz 
 gr\'afica b\'asica, sin embargo, es suficiente, ya que la intenci\'on es 
 proveer de un sistema de f\'acil uso y reducir el n\'umero de pasos 
 necesarios para realizar una tarea en el equipo de c\'omputo.


La metodolog\'ia mencionada, hace uso de un grafo dirigido el cual se va 
 construyendo mientras el usuario hace uso de la computadora y en el momento 
 se le muestran las secuencias encontradas, para que \'el identifique si le 
 es \'util o no. Las acciones definidas como \'utiles, son registradas para 
 su posterior uso, permitiendo que el usuario haga uso de estas, mejorando 
 su rendimiento y las sugerencias de tareas que se le muestran \'el.


Considerando el objetivo de la metodolog\'ia en este trabajo, se clasifica 
 como una variante de Robotic Process Automation (RPA).
