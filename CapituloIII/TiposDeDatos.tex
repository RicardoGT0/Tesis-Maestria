\section{Tipos de datos}

En minería de datos existen 2 clasificaciones de datos[Discretization Methods], los cualitativos(por ejemplo: tipo de sangre, opinión personal, código postal, teléfono) y numéricos. A continuación se presentan sus características.

\begin{itemize}
\item Cualitativos o categóricos 
	\begin{itemize}
	\item Son lo que se pueden ubicar en distintas  categorías.
	\item Algunos se pueden ordenar en orden significativo
	\item No se le puede aplicar operaciones matemáticas
	\end{itemize}
	 

\item Cuantitativos
	\begin{itemize}
	\item Son de naturaleza numérica. 
	\item Se pueden clasificar en orden. 
	\item Admiten operaciones aritméticas significativas. 
	\item Pueden ser discretos o continuos.		
	\end{itemize}
\end{itemize}

Los datos también se pueden clasificar por la forma en que se categorizan, cuentan o miden. Este tipo de clasificación utiliza escalas de medición, y cuatro niveles comunes de escalas son: 

\begin{itemize}
\item Nominal:
	\begin{itemize}
	\item Se clasifican los datos en categorías mutuamente excluyentes (no superpuestas) y exhaustivas en las que no se puede imponer un orden o clasificación significativa en los datos. 
	\end{itemize}
		
\item Ordinal:
	\begin{itemize}
	\item Se clasifican los datos en categorías que se pueden clasificar. 
	\item Las diferencias entre los rangos no pueden calcularse mediante aritmética.
	\item Se pueden ordenar
	\end{itemize}		

\item Intervalo: 
	\begin{itemize}
	\item Se clasifican los datos y las diferencias entre las unidades de medida se pueden calcular mediante aritmética. 
	\item El cero en el nivel de intervalo de medición no significa 'null' o 'nada' como cero en aritmética.
	\end{itemize}		

\item Relación
	\begin{itemize}
	\item Se clasifican los datos y las diferencias entre las unidades de medida se pueden calcular mediante aritmética. 
	\item El cero en el nivel de intervalo de medición significa 'null' o 'nada' como cero en aritmética.
	\end{itemize}
		
\end{itemize}