\section{Teoría de árboles}
En términos matemáticos un árbol es un grafo $T$ el cual se puede definir de
 la siguiente forma\cite{SUSANNAS.EPP2012}:

\begin{itemize}
	\item ``Un grafo se llama árbol si y solo si, está libre de circuitos y
	 es conexo.''
	\item ``Un vértice de grado 1 en $T$ se denomina un vértice terminal (o
	 una hoja).''
	\item ``Un vértice de grado superior a 1 en $T$ es un vértice interno (o
	 un vértice de rama).''
\end{itemize}

Los arboles son muy usados en la actualidad ya que permiten  darle sentido a
 la información contenida, gracias a la asociatividad, parentización y
 prioridad que este permite de manera implícita. Entre sus usos múltiples en
 el área de la informática se pueden destacar los  siguientes; relaciones
 entre módulos de programación, arboles de decisión en inteligencia artificial
 y representaciones gramáticas\cite{gutierrez1999estructuras}.  

Una forma de ver a un árbol puede ser como un solo nodo, esté recibe el nombre
 de  nodo raíz, al cual se le puede enraizar un sinfín de arboles lo que da
 origen  a un árbol con otras características, dependiendo del resultado se le
 puede clasificar en alguno de los modelos existentes, por ejemplo; árbol 
 general o árbol binario\cite{gutierrez1999estructuras}. 

El árbol general es un modelo con una cantidad indeterminada de nodos hijos,
 mientras que el árbol binario es un caso particular del árbol general ya que
 este tiene la característica de tener siempre en cada nodo 2 nodos hijos como
 máximo, cabe mencionar que este es uno de los modelos a mas usados, así como
 el ultimo caso mencionado, hay arboles que tienen una cantidad fija de nodos
 hijos, de manera general estos son llamados arboles n-arios
 \cite{gutierrez1999estructuras}.