Desde el Microsoft Windows $3.11$ \cite{RomeroZunica1998} hasta la actualidad las 
 Opciones de accesibilidad de Microsoft Windows han permitido que el
 sistema operativo sea posible usarlo sin importar las condiciones f\'isicas del
 usuario \cite{DanielHubbell2016}. A continuaci\'on, se menciona una breve 
 descripci\'on de estas opciones.

\begin{itemize}
	\item Lupa: Aumenta el tama\~no del contenido de la pantalla para que sea
	 m\'as f\'acil leerlo \cite{xatakaaccesiblilidad}.
	\item Narrador:  Esta funci\'on es una ayuda auditiva que ayuda a saber 
	 cu\'ales son las ventanas abiertas y su contenido, por medio de una voz
	 sintetizada \cite{xatakaaccesiblilidad}.
	\item Teclado en pantalla: Como su nombre lo indica es un teclado virtual
	 con el cual podemos escribir presionando la pantalla, en caso de ser un
	 dispositivo t\'actil, o presionando los botones con el rat\'on
	 \cite{xatakaaccesiblilidad}.
	\item Contraste alto: Otra ayuda visual para poder distinguir mejor los
	 elementos en pantalla modificando el contraste de Windows
	 \cite{xatakaaccesiblilidad}.
	\item Reconocimiento de voz: Es una herramienta de dictado con la cual
	 se puede escribir y manipular aplicaciones e incluso el mismo sistema
	 operativo por medio de comandos de voz \cite{support14213}.
\end{itemize}

Aunque no esta considerado dentro de las opciones de accesibilidad de Windows,
 el asistente \emph{Microsoft Cortana} facilita la realizaci\'on de algunas tareas por medio de
 comandos de voz, por ejemplo, apertura de programas, la manipulaci\'on de
 recordatorios, mensajes de texto y correo electr\'onico \cite{support17214}. 