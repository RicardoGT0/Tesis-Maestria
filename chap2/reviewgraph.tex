
En \cite{QIAO2018336}, L. Qiao et al. presenta una revisi\'on de 
 metodolog\'ias cuyo objetivo es analizar las propiedades valiosas o minar 
 patrones informativos en grafos, ya que en los problemas con; espacios de 
 alta dimensionalidad, con diferentes distribuciones, con ruido en los 
 datos o incertidumbre, se necesitan construir grafos antes de cualquier 
 an\'alisis o tarea de aprendizaje. 


Para la construcci\'on de un grafo a partir de un conjunto de v\'ertices o 
 nodos \cite{QIAO2018336}, primero se determina la estructura topol\'ogica 
 del grafo, para lo cual se pueden hacer uso de las siguientes 
 metodolog\'ias que se dividen en tres grupos; el primero son los 
 algoritmos libres de par\'ametros, los cuales son, \'arbol de expansi\'on 
 m\'inimo (minimum spanning tree, MST), triangulaci\'on de Delaunay 
 (Delaunay triangulation, DT), grafo de Gabriel (Gabriel graph, GG) y 
 gr\'afico de vecindad relativa (Relative neighborhood graph, RNG).  El 
 segundo grupo incluye esqueleto beta (Beta-skeleton), K vecinos m\'as 
 cercanos (K nearest neighbors, KNN), vecindario de la bola-$\varepsilon$ 
 ($\varepsilon$-ball neighborhood, $\varepsilon$-N) y concordancia B 
 (B-matching) y en el tercer grupo se incluyen aquellos con par\'ametros 
 flexibles.


Despu\'es considerando el conjunto de bordes \cite{QIAO2018336} o aristas 
 obtenido se determina la matriz de pesos por una metodolog\'ia de las 
 mencionadas en las siguientes dos categor\'ias. La primera es por 
 definici\'on directa entre las cuales est\'an; peso binario (Binary 
 weight), Kernel Gaussiano (Gaussian kernel) y sus variantes, inversa de 
 distancia (Inverse of distance), correlaci\'on completa (Full correlation) 
 y correlaci\'on parcial (Partial correlation). 


La otra categor\'ia es, aprendizaje autom\'atico \cite{QIAO2018336} en la 
 cual se encuentran englobadas las siguientes siete metodolog\'ias; 
 aprendizaje de grafos param\'etricos (por ejemplo, ponderaci\'on de borde 
 adaptativa (Adaptive Edge Weighting, AEW)), aprendizaje de grafos 
 induciendo la localidad (por ejemplo, Reconstrucci\'on lineal local (Local 
 linear reconstruction, LLR)), aprendizaje de grafos basado en 
 reconstrucci\'on global (por ejemplo, ajuste de grafico a vector (Fitting 
 a Graph to Vector, FGV)), aprendizaje de grafos induciendo la escasez (por 
 ejemplo, grafo $L_1$), Aprendizaje de grafos de bajo rango (por ejemplo, 
 Markov random walk (MRW)), Aprendizaje mult\'igrafo y finalmente el 
 aprendizaje de grafos en un espacio transformado adaptable (por ejemplo, 
 Modelo de conservaci\'on de la localidad optimizado para grafos (graph-
 optimized locality preserving model, GoLP)). 

