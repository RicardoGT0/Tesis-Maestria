
Nuestra propuesta, a diferencia de las anteriores; obtiene m\'ultiples tareas
 que se podr\'ian considerar tareas objetivo, la metodolog\'ia implementada 
 es relativamente simple ya que en esencia es un grafo dirigido, y provee de 
 interacci\'on con todos los programas que el usuario requiera sin realizar 
 cambio al software. En la tabla \ref{ComparativaT} se presenta la 
 comparativa de estas caracter\'isticas, se observa que los que tienen mayor 
 similitud son, \emph{Microsoft Cortana}, un \emph{archivo por lotes}, 
 \emph{UIPath (o RPA)} y el \emph{Pulover\textsc{\char13}s Macro Creator},
 sin embargo, considerando estos requisitos, se tienen que considerar las 
 siguientes diferencias respecto a la propuesta planteada:

\begin{itemize}

\item {La propuesta, va a encontrar las tareas objetivo, mientras que las
 dem\'as metodolog\'ias y sistemas, se les debe indicar cual es la tarea
 objetivo.}

\item {La propuesta va a ser gen\'erica, ya que esta va enfocada a un sector
 de la poblaci\'on donde se puede hacer uso cualquier tipo de software, sin
 embargo, para la mayor\'ia de los trabajos mencionados, se debe conocer de
 antemano el software con el cual va a interactuar el usuario, a excepci\'on
 de UIPath y Pulover\textsc{\char13}s Macro Creator, ya que son sistemas
 gen\'ericos que permiten ser configurados.}

\item {La propuesta debe ser de f\'acil uso, la intensi\'on es facilitar el
 uso de la computadora, por lo que la persona deber\'ia interactuar lo
 m\'inimo posible con el software desarrollado. La mayor\'ia de los trabajos 
 investigados, requieren de un experto en el sistema de automatizaci\'on para 
 que este sea configurado correctamente, Pulover\textsc{\char13}s Macro 
 Creator, permite grabar la secuencia de acciones para llevar acabo la tarea 
 y posteriormente reproducirla, sin embargo, esto solo aplica a tareas 
 sencillas, ya que el se requiere de un conocimiento mas profundo del 
 software para realizar la edici\'on de la secuencia grabada o incluso la 
 creaci\'on de una.}

\end{itemize}


Por lo mencionado anteriormente, se determina que el desarrollo expuesto en
 la presente tesis es una variante o una posible mejora del RPA y los
 generadores de macros.


\begin{table}[!h]
\centering
\scalebox{0.75}{
\begin{tabular}{|p{3.5cm} || p{3.5cm} | p{1.75cm} | p{3.5cm} | p{2.25cm} | p{2cm}|}
\hline

\textbf{Nombre del autor o del proyecto}
& \textbf{interacci\'on con todos los programas del usuario}
& \textbf{Facilidad de uso}
& \textbf{Metodolog\'ia empleada}
& \textbf{Automatizar acciones}
& \textbf{M\'ultiples tareas objetivo}
\\ 
\hline
\hline

Microsoft Cortana
& Solo de Microsoft
& Si
& Patentado
& Si
& Si
\\ 
\hline

Archivo por lotes
& Solo el Sistema Operativo
& Si
& Scripts
& Si
& Si
\\ 
\hline

Pulover\textsc{\char13}s Macro Creator
& Si
& Si
& Scripts
& Si
& Si
\\ 
\hline

A. Nakano, et al.
& No
& Si
& No especifica
& Si
& No
\\ 
\hline

ANFIS
& No
& Si
& Sistema de Inferencia Neurodifuso Adaptativo
& Si
& No
\\ 
\hline

Shogo Nishiguchi, et al.
& No
& Si
& Interprete
& No
& Si
\\ 
\hline

SPARC
& No
& Si
& Aprendizaje por refuerzo 
& Si
& No
\\ 
\hline

VAHM3
& No
& Si
& algoritmo de condensaci\'on
& Si
& No
\\ 
\hline

UIPath
& Si
& Si
& Robotic Process Automation
& Si
& Si
\\ 
\hline

\textbf{Propuesta}
& Si
& Si
& Grafo dirigido
& Si
& Si
\\ 
\hline

\end{tabular}
}
\caption{Tabla de requisitos y an\'alisis comparativo}
\label{ComparativaT}
\end{table}