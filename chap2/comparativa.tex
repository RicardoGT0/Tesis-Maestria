
Nuestra propuesta, a diferencia de las anteriores; obtiene m\'ultiples tareas
 que se podr\'ian considerar objetivo, la metodolog\'ia implementada es 
 relativamente simple ya que en esencia es un grafo dirigido, y provee de 
 interacci\'on con todos los programas que el usuario requiera sin realizar 
 cambio al software. En la tabla \ref{ComparativaT} se presenta la comparativa 
 de estas caracter\'isticas, se observa que los que tienen mayor similitud son, 
 \emph{Microsoft Cortana}, un archivo por lotes y el Pulover\textsc{\char13}s
 Macro Creator.


\begin{table}[!h]
\centering
\scalebox{0.75}{
\begin{tabular}{|p{3.5cm} | p{3.5cm} | p{2cm} | p{3.5cm} | p{2cm} | p{2cm}|}
\hline

Nombre del autor o del proyecto
& interacci\'on con todos los programas del usuario
& Facilidad de uso
& Metodolog\'ia empleada
& Automatizar acciones
& Objetivo a m\'ultiples tareas
\\ 
\hline

Microsoft Cortana
& Solo de Microsoft
& Si
& Patentado
& Si
& Si
\\ 
\hline

Archivo por lotes
& Solo el Sistema Operativo
& Si
& Scripts
& Si
& Si
\\ 
\hline

Pulover\textsc{\char13}s Macro Creator
& Si
& Si
& Scripts
& Si
& Si
\\ 
\hline

A. Nakano, et al.
& No
& Si
& No especifica
& Si
& No
\\ 
\hline

ANFIS
& No
& Si
& Sistema de Inferencia Neurodifuso Adaptativo
& Si
& No
\\ 
\hline

Shogo Nishiguchi, et al.
& No
& Si
& Interprete
& No
& Si
\\ 
\hline

SPARC
& No
& Si
& Aprendizaje por refuerzo 
& Si
& No
\\ 
\hline

VAHM3
& No
& Si
& algoritmo de condensaci\'on
& Si
& No
\\ 
\hline

Propuesta
& Si
& Si
& Grafo dirigido
& si
& si
\\ 
\hline

\end{tabular}
}
\caption{Tabla de requisitos y an\'alisis comparativo}
\label{ComparativaT}
\end{table}