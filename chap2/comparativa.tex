El software que ayude a solucionar el problema planteado; debe de tener
 interacci\'on con todos los programas del usuario, sin importar cual sea, 
 dado que se desconoce el software que \'el pueda llegar a ocupar. Debe ser 
 de f\'acil uso, considerando el grupo de personas a las que va dirigido, 
 adem\'as, la intenci\'on es agilizar las tareas realizadas, no el darle mas 
 trabajo al usuario. El software debe de tener la capacidad de automatizar 
 varias tareas, mientras m\'as se automaticen, mejor. Tomando en 
 consideraci\'on estos requisitos, se realiz\'o la tabla \ref{ComparativaT} 
 en la que se someten a evaluaci\'on los proyectos ya presentados en este 
 cap\'itulo. 

\begin{table}[!h]
\centering
\caption{Tabla de requisitos y an\'alisis comparativo}
\scalebox{0.7}{
\begin{tabular}{|p{3.5cm} || p{3.5cm} | p{1.75cm} | p{3.5cm} | p{2.25cm} | p{2cm}|}
\hline

\textbf{Nombre del autor o del proyecto}
& \textbf{interacci\'on con todos los programas del usuario}
& \textbf{Facilidad de uso}
& \textbf{Metodolog\'ia empleada}
& \textbf{Automatizar acciones}
& \textbf{M\'ultiples tareas objetivo}
\\ 
\hline
\hline

Microsoft Cortana
& Solo de Microsoft
& Si
& Patentado
& Si
& Si
\\ 
\hline

Archivo por lotes
& Solo el Sistema Operativo
& Si
& Scripts
& Si
& Si
\\ 
\hline

Pulover\textsc{\char13}s Macro Creator
& Si
& Si
& Scripts
& Si
& Si
\\ 
\hline

A. Nakano, et al.
& No
& Si
& No especifica
& Si
& No
\\ 
\hline

ANFIS
& No
& Si
& Sistema de Inferencia Neurodifuso Adaptativo
& Si
& No
\\ 
\hline

Shogo Nishiguchi, et al.
& No
& Si
& Interprete
& No
& Si
\\ 
\hline

SPARC
& No
& Si
& Aprendizaje por refuerzo 
& Si
& No
\\ 
\hline

VAHM3
& No
& Si
& algoritmo de condensaci\'on
& Si
& No
\\ 
\hline

UIPath
& Si
& Si
& Robotic Process Automation
& Si
& Si
\\ 
\hline

\textbf{\emph{Propuesta}}
& \textbf{\emph{Si}}
& \textbf{\emph{Si}}
& \textbf{\emph{Grafo}}
& \textbf{\emph{Si}}
& \textbf{\emph{Si}}
\\ 
\hline

\end{tabular}
}

\label{ComparativaT}
\end{table}

Empezando por Microsoft Cortana, su uso se expande a oraciones habladas o 
 escritas aceptando lenguaje coloquial, pero con ordenes limitadas a la 
 interacci\'on con el software de la misma compa\~n\'ia, sin embargo, se le 
 pueden dar \'ordenes para agendar reuniones o eventos, e incluso enviar 
 mensajes de correo electr\'onico o texto con comandos de voz. 


Un archivo por lotes permite la automatizaci\'on de rutinas repetitivas que
 el usuario realiza en el sistema operativo, por ejemplo, revisar si hay 
 actualizaciones disponibles y abrir alg\'un archivo en espec\'ifico, pero 
 para esto es necesario que inicialmente el usuario o alg\'un experto 
 programe el script, posteriormente el usuario solo debe ejecutar dicho 
 script y se realizaran las tareas programadas.


En el creador de macros, como Pulover\textsc{\char13}s Macro Creator, se 
 tiene una interfaz y una grabadora de macros que facilitan la creaci\'on de 
 un script, incluso este es posible que manipule aplicaciones de distintitos 
 desarrolladores, as\'i como la operaci\'on de ventanas en segundo plano, 
 pero el desarrollo del script m\'as all\'a de lo entregado por la grabadora, 
 requiere un amplio conocimiento por parte del usuario o incluso de un 
 experto, claro est\'a, una vez desarrollado el script el uso de este, es 
 f\'acil.


Otro software que permite la automatizaci\'on de tareas repetitivas es 
 UIPath, el cual utiliza la metodolog\'ia de RPA, para automatizar procesos 
 realizados sobre cualquier software, permitiendo la manipulaci\'on de 
 aplicaciones sin importar el desarrollador, sin embargo, al igual que el 
 creador de macros, se requiere de un experto en el software para la 
 automatizaci\'on de procesos completos. 


Nuestra propuesta, a diferencia de las anteriores; obtiene m\'ultiples tareas
 que se podr\'ian considerar tareas objetivo, la metodolog\'ia implementada 
 es relativamente simple ya que en esencia es un grafo dirigido, y provee de 
 interacci\'on con todos los programas que el usuario requiera sin realizar 
 cambio al software. En la tabla \ref{ComparativaT}, se observa que los 
 proyectos que tienen mayor 
 similitud son, \emph{Microsoft Cortana}, un \emph{archivo por lotes}, 
 \emph{UIPath (o RPA)} y el \emph{Pulover\textsc{\char13}s Macro Creator},
 sin embargo, considerando estos requisitos, se tienen que considerar las 
 siguientes diferencias respecto a la propuesta planteada:

\begin{itemize}

\item {La propuesta, va a encontrar las tareas objetivo, mientras que a las
 dem\'as metodolog\'ias y sistemas, se les debe indicar cual es la tarea
 objetivo.}

\item {La propuesta va a ser gen\'erica, ya que esta va enfocada a un sector
 de la poblaci\'on donde se puede hacer uso de cualquier tipo de software, sin
 embargo, para la mayor\'ia de los trabajos mencionados, se debe conocer de
 antemano el software con el cual va a interactuar el usuario, a excepci\'on
 de UIPath y Pulover\textsc{\char13}s Macro Creator, ya que son sistemas
 gen\'ericos que permiten ser configurados.}

\newpage

\item {La propuesta debe ser de f\'acil uso, la intensi\'on es facilitar el
 uso de la computadora, por lo que la persona deber\'ia interactuar lo
 m\'inimo posible con el software desarrollado. La mayor\'ia de los trabajos 
 investigados, requieren de un experto en el sistema de automatizaci\'on para 
 que este sea configurado correctamente, Pulover\textsc{\char13}s Macro 
 Creator, permite grabar la secuencia de acciones para llevar acabo la tarea 
 y posteriormente reproducirla, sin embargo, esto solo aplica a tareas 
 sencillas, ya que se requiere de un conocimiento mas profundo del 
 software para realizar la edici\'on de la secuencia grabada o incluso la 
 creaci\'on de una.}

\end{itemize}


Por lo mencionado anteriormente, se determina que el desarrollo expuesto en
 la presente tesis es una alternativa o una posible mejora del RPA y los
 creadores de macros.
