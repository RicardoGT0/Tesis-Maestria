
De la l\'inea 1 a la 17 se define el inicio del \emph{Procedimiento de 
 monitoreo}, en este se leen los dispositivos de entrada, se genera el grafo 
 y se buscan las secuencias. Este proceso, es necesario que permanezca activo 
 desde que se enciende la computadora hasta que se apague el equipo, para 
 capturar cada acci\'on que realice el usuario, por lo que en la l\'inea 2 se 
 inicia un ciclo \emph{Mientras} o \emph{While} que no tiene condici\'on de 
 paro y abarca hasta la l\'inea 16. 

\begin{tiny}
\begin{lstlisting}[name=EXmonitor][firstnumber=1]
Procedimiento monitoreo
Mientras Verdad Hacer
\end{lstlisting}
\end{tiny}

%\noindent
En la l\'inea 3, se lee la acci\'on del dispositivo de entrada y esta es 
 almacenada en forma de \emph{Nodo}.


\begin{tiny}
\begin{lstlisting}[name=EXmonitor][firstnumber=last]
    Nodo = Capturar accion de teclado o raton
\end{lstlisting}
\end{tiny}

En la l\'inea 4, se verifica que exista un \emph{Nodo} con la misma 
 informaci\'on que el \emph{Nodo} reci\'en creado en el \emph{Grafo}, en caso 
 de ser as\'i, al \emph{Nodo} existente se le aumenta 1 al 
 \emph{contador\_Nodo}, operaci\'on realizada en la l\'inea 5, 
 adicionalmente, se va a empezar con una secuencia de validaciones que 
 servir\'an para identificar una secuencia y posteriormente una tarea. 

\begin{tiny}
\begin{lstlisting}[name=EXmonitor][firstnumber=last]
    Si Nodo existe en Grafo Entonces
        Incrementar contador_Nodo
\end{lstlisting}
\end{tiny}

Primero, se verifica que el \emph{Nodo} se haya repetido mas de 70 veces.

\begin{tiny}
\begin{lstlisting}[name=EXmonitor][firstnumber=last]
        Si contador_Nodo > 70 Entonces
\end{lstlisting}
\end{tiny}

En el caso de ser verdadera la verificaci\'on de la l\'inea 6, se verifica 
 que el \emph{Nodo} exista en una lista de nodos llamada \emph{Secuencia}, 
 esta es una tarea que ha realizado el usuario.

\begin{tiny}
\begin{lstlisting}[name=EXmonitor][firstnumber=last]
            Si Nodo existe en Secuencia Entonces
\end{lstlisting}
\end{tiny}

En el caso de ser verdadera la verificaci\'on de la l\'inea 7, en la l\'inea
 8, se ejecuta el procedimiento \emph{Seleccion\_Tarea}, en el cual se va  
 identificar si la \emph{Secuencia} actual es una \emph{Tarea} que le pueda 
 ser de utilidad al usuario. 

\begin{tiny}
\begin{lstlisting}[name=EXmonitor][firstnumber=last]         
                Ejecutar Seleccion_Tarea
\end{lstlisting}
\end{tiny}

Despu\'es de verificar si la secuencia es una \emph{Tarea} o no, la l\'inea
 \emph{Borra} la \emph{Secuencia} dej\'andola sin \emph{Nodos}, para empezar a
 procesar con otra \emph{Secuencia}.

\begin{tiny}
\begin{lstlisting}[name=EXmonitor][firstnumber=last]         
                Borrar Secuencia
\end{lstlisting}
\end{tiny}

La instrucci\'on 10 es el caso contrario de la verificaci\'on de la l\'inea
 7, por tal, en caso de que el \emph{Nodo} no se encuentre en la 
 \emph{Secuencia}, este se agregar\'a.

\begin{tiny}
\begin{lstlisting}[name=EXmonitor][firstnumber=last]        
            Si no Entonces
               Agregar nodo a Secuencia
\end{lstlisting}
\end{tiny}

La l\'inea 13 es para el caso falso de la verificaci\'on de la l\'inea 6, 
 como el \emph{Nodo} no cumple la condici\'on de tener mas de 70 
 repeticiones, por tal, se procesa la \emph{Secuencia} hasta el momento en
  busca de \emph{Tareas} (l\'inea 13)y porteriormente, se \emph{Borra} la 
  \emph{Secuencia} (l\'inea 14).

\begin{tiny}
\begin{lstlisting}[name=EXmonitor][firstnumber=last]
        Si no Entonces
            Ejecutar Seleccion_Tarea
            Borrar Secuencia
\end{lstlisting}
\end{tiny}


En la l\'inea 15, se presenta el caso contrario de la (l\'inea 4), que el 
 \emph{Nodo} no exista en el grafo, este se agrega (l\'inea 16) en el 
 \emph{Grafo}, enlaz\'andolo desde el \emph{Nodo} anterior. 


\begin{tiny}
\begin{lstlisting}[name=EXmonitor][firstnumber=last]
    Si no Entonces
        Agregar Nodo en Grafo
\end{lstlisting}
\end{tiny}


Por ultimo; en la l\'inea 17 se encuentra el final del \emph{Mientras} que se 
 inici\'o en la l\'inea 2, y en la l\'inea 18 est\'a el final del 
 \emph{Procedimiento monitoreo} que se empez\'o en la l\'inea 1.

\begin{tiny}
\begin{lstlisting}[name=EXmonitor][firstnumber=last]
Fin Mientras
Fin Procedimiento 
\end{lstlisting}
\end{tiny}




En las l\'ineas siguientes, se presenta el pseudo--codigo del procedimiento
 para la identificaci\'on de tareas. Empezando nuevamente por la l\'inea 1, se 
 presenta la el inicio del mismo, el cual comprende de 14 l\'ineas. Al ser un 
 m\'odulo del procedimiento anterior, este se trabaja con la misma 
 \emph{Secuencia} que el anterior, por tal, en la l\'inea 2, se procede a 
 verificar que la \emph{Secuencia} exista en la \emph{lista\_Secuencias} 

\begin{tiny}
\begin{lstlisting}[name=EXseleccion][firstnumber=1]
Procedimiento Seleccion_Tarea
Si Secuencia existe en lista_Secuencias Entonces
\end{lstlisting}
\end{tiny}

Si la verificaci\'on de la l\'inea 2 es verdadera, se verifica que la 
 \emph{Secuencia} se haya repetido mas de 5 veces, para validar que le pueda 
 ser de utilidad al usuario, adem\'as se verifica que la \emph{secuencia} no 
 exista en la \emph{lista\_Tareas} o en la \emph{lista\_Ignoradas}, ya que 
 esto significar\'ia que ha sido encontrada anteriormente. 
 En caso de que esta validaci\'on de la l\'inea 3 resulte verdadera
 y que el usuario considere que s\'i le es \'util, en la l\'inea 4, se le
 solicita que le asigne un nombre a la \emph{Secuencia} mostrada. 
 Posteriormente en la l\'inea 5 se lee la \emph{Respuesta} introducida por el 
 usuario.

\begin{tiny}
\begin{lstlisting}[name=EXseleccion][firstnumber=last]
    Si contador_Secuencia > 5 Y Secuencia No existe en lista_Tareas O en lista_Ignoradas Entonces
        Escribir ``Si desea Guardar la tarea [Secuencia], escriba un nombre''
        Leer Respuesta
\end{lstlisting}
\end{tiny}

En caso de que la \emph{respuesta} sea un nombre para la acci\'on se ejecuta 
 la l\'inea 7, en la que se agrega la \emph{Secuencia} a una 
 \emph{lista\_Tareas}, que son las \emph{Tareas} a las que tiene acceso el 
 usuario, para poder ejecutarlas. En caso contrario se prosigue con la 
 l\'inea 4, agregando la \emph{Secuencia} a la \emph{lista\_Ignoradas}, 
 estas son las tareas que el usuario ha decidido que no le son de utilidad y 
 por tal no se le volver\'an a mostrar.

\begin{tiny}
\begin{lstlisting}[name=EXseleccion][firstnumber=last]
        Si Respuesta es nombre Entonces
            Agregar Secuencia a lista_Tareas
        Si no Entonces 
            Agregar Secuencia a lista_Ignoradas
\end{lstlisting}
\end{tiny}

En caso contrario de que la verificaci\'on de la l\'inea 3 no sea verdadera, 
 en la l\'inea 11, se incrementa en 1 el \emph{contador\_Secuencia}, para 
 indicar que se ha usado una vez m\'as la \emph{Secuencia}.

\begin{tiny}
\begin{lstlisting}[name=EXseleccion][firstnumber=last]
    Si no Entonces
        Incrementar contador_Secuencia
\end{lstlisting}
\end{tiny}

Para el caso de que la verificaci\'on de la l\'inea 2 sea falsa, se prosigue 
 con la l\'inea 13, agregando la \emph{Secuencia} a la 
 \emph{lista\_Secuencias}.

\begin{tiny}
\begin{lstlisting}[name=EXseleccion][firstnumber=last]
Si no Entonces
    Agregar Secuencia a Lista_Secuencias
\end{lstlisting}
\end{tiny}

Finalmente, se presenta en la l\'inea 14, el f\'in del procedimiento
 \emph{Seleccion\_Tarea}, en este punto se tienen tres posibilidades para la 
 \emph{Secuencia} actual; primero, la \emph{Secuencia} puede haber sido de 
 utilidad para el usuario, por lo que se guardo en \emph{lista\_Tareas} con el 
 nombre asignado como referencia, el segundo caso es, que la \emph{Secuencia} 
 no le fue de utilidad al usuario y fue almacena en \emph{lista\_Ignoradas} y 
 en el \'ultimo caso la \emph{Secuencia} no cumpl\'ia la caracter\'istica de 
 las 5 repeticiones y solo se queda almacenada en \emph{lista\_Secuencias}  

\begin{tiny}
\begin{lstlisting}[name=EXseleccion][firstnumber=last]
Fin Procedimiento
\end{lstlisting}
\end{tiny}