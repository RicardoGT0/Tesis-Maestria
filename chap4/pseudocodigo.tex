
A continuaci\'on se presenta el pseudo--codigo del programa desarrollado.


\begin{table}[]
\begin{tiny}
\begin{lstlisting}[style=C]
Procedimiento monitoreo
Mientras Verdad Hacer

\end{lstlisting}
\end{tiny}
\end{table}

\begin{table}[]
\begin{tiny}
\begin{lstlisting}[style=C]
    Nodo = Capturar accion de teclado o raton
    Si Nodo existe en grafo Entonces
        Incrementar contador_Nodo

\end{lstlisting}
\end{tiny}
\end{table}

\begin{table}[]
\begin{tiny}
\begin{lstlisting}[style=C]
    Si no Entonces
        Colocar Nodo en grafo

\end{lstlisting}
\end{tiny}
\end{table}


\begin{table}[]
\begin{tiny}
\begin{lstlisting}[style=C]
        Si contador_nodo > 70 Entonces

\end{lstlisting}
\end{tiny}
\end{table}

        
\begin{table}[]
\begin{tiny}
\begin{lstlisting}[style=C]
            Si Nodo existe en Secuencia Entonces

\end{lstlisting}
\end{tiny}
\end{table}


\begin{table}[]
\begin{tiny}
\begin{lstlisting}[style=C]
                Si Secuencia existe en Lista_secuencias Entonces

\end{lstlisting}
\end{tiny}
\end{table}


\begin{table}[]
\begin{tiny}
\begin{lstlisting}[style=C]
                    Si contador_Secuencia > 5 Entonces
                        Escribir ``Si desea Guardar la tarea [Secuencia], escriba un nombre''
                        Leer Respuesta

\end{lstlisting}
\end{tiny}
\end{table}                        
                        
\begin{table}[]
\begin{tiny}
\begin{lstlisting}[style=C]  
                        Si Respuesta es nombre Entonces
                            Agregar Secuencia a Lista_Tareas
                        Si no Entonces 
                            Agregar Secuencia a Lista_Ignoradas

\end{lstlisting}
\end{tiny}
\end{table}


\begin{table}[]
\begin{tiny}
\begin{lstlisting}[style=C]
                    Si no Entonces
                    Incrementar contador_Secuencia

\end{lstlisting}
\end{tiny}
\end{table}
                    
                    
\begin{table}[]
\begin{tiny}
\begin{lstlisting}[style=C]
                Si no Entonces
                    Agregar Secuencia a Lista_Secuencias

\end{lstlisting}
\end{tiny}
\end{table}
            

\begin{table}[]
\begin{tiny}
\begin{lstlisting}[style=C]            
            Borrar Secuencia

\end{lstlisting}
\end{tiny}
\end{table}
            
            
\begin{table}[]
\begin{tiny}
\begin{lstlisting}[style=C]            
            Si no Entonces
                Agregar nodo a Secuencia

\end{lstlisting}
\end{tiny}
\end{table}
		
		
\begin{table}[]
\begin{tiny}
\begin{lstlisting}[style=C]		
		Si no Entonces
            Borrar Secuencia
Fin Mientras
Fin Procedimiento 

\end{lstlisting}
\end{tiny}
\end{table}


\begin{table}[]
\begin{tiny}
\begin{lstlisting}[style=C]
Procedimiento Ejecucion
Mientras Verdad Hacer

\end{lstlisting}
\end{tiny}
\end{table}


\begin{table}[]
\begin{tiny}
\begin{lstlisting}[style=C]
Escribir ``Selecciona la tarea a ejecutar''
Leer Seleccion
Escribir ``Ejecutar Accion seleccionada?''
Leer Decisi\'on

\end{lstlisting}
\end{tiny}
\end{table}


\begin{table}[]
\begin{tiny}
\begin{lstlisting}[style=C]
Si Decision = ``Si'' Entonces
	Ejecutar Seleccion

\end{lstlisting}
\end{tiny}
\end{table}


\begin{table}[]
\begin{tiny}
\begin{lstlisting}[style=C]
Fin Mientras
Fin Procedimiento

\end{lstlisting}
\end{tiny}
\end{table}
