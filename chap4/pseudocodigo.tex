
A continuaci\'on se presenta el pseudo--codigo del programa desarrollado, el
 cual consta de 2 partes que se ejecutan de manera simultanea, el 
 \emph{Procedimiento de monitoreo} y el \emph{Procedimiento de ejecuci\'on}.

De la l\'inea 1 a la 28 se define el inicio del \emph{Procedimiento de monitoreo}, en este se leen los dispositivos de entrada, se genera el grafo y se buscan las secuencias. Este proceso, es necesario que permanezca activo desde que se enciende la computadora hasta que se apague el equipo, para capturar cada acci\'on que realice el usuario, por lo que en la l\'inea 2 se inicia un ciclo \emph{Mientras} o \emph{While} que no tiene condici\'on de paro y abarca hasta la l\'inea 27. 

\begin{tiny}
\begin{lstlisting}[style=C]
Procedimiento monitoreo
Mientras Verdad Hacer
\end{lstlisting}
\end{tiny}

%\noindent
En la l\'inea 3, se le\'e la acci\'on del dispositivo de entrada y esta es almacenada en forma de \emph{Nodo}.


\begin{tiny}
\begin{lstlisting}[style=C]
    Nodo = Capturar accion de teclado o raton
\end{lstlisting}
\end{tiny}

En la l\'inea 4, se verifica que exista un \emph{Nodo} con la misma informaci\'on que el \emph{Nodo} reci\'en creado en el grafo, en caso de ser as\'i, al \emph{Nodo} existente se le aumenta el \emph{contador\_Nodo} en la unidad, acci\'on realizada en la l\'inea 5.

\begin{tiny}
\begin{lstlisting}[style=C]
    Si Nodo existe en grafo Entonces
        Incrementar contador_Nodo
\end{lstlisting}
\end{tiny}





\begin{tiny}
\begin{lstlisting}[style=C]
        Si contador_nodo > 70 Entonces
\end{lstlisting}
\end{tiny}



\begin{tiny}
\begin{lstlisting}[style=C]
            Si Nodo existe en Secuencia Entonces
\end{lstlisting}
\end{tiny}



\begin{tiny}
\begin{lstlisting}[style=C]
                Si Secuencia existe en Lista_secuencias Entonces
\end{lstlisting}
\end{tiny}



\begin{tiny}
\begin{lstlisting}[style=C]
                    Si contador_Secuencia > 5 Entonces
                        Escribir ``Si desea Guardar la tarea [Secuencia], escriba un nombre''
                        Leer Respuesta
\end{lstlisting}
\end{tiny}



\begin{tiny}
\begin{lstlisting}[style=C]  
                        Si Respuesta es nombre Entonces
                            Agregar Secuencia a Lista_Tareas
                        Si no Entonces 
                            Agregar Secuencia a Lista_Ignoradas
\end{lstlisting}
\end{tiny}



\begin{tiny}
\begin{lstlisting}[style=C]
                    Si no Entonces
                    Incrementar contador_Secuencia
\end{lstlisting}
\end{tiny}



\begin{tiny}
\begin{lstlisting}[style=C]
                Si no Entonces
                    Agregar Secuencia a Lista_Secuencias
\end{lstlisting}
\end{tiny}



\begin{tiny}
\begin{lstlisting}[style=C]            
            Borrar Secuencia
\end{lstlisting}
\end{tiny}



\begin{tiny}
\begin{lstlisting}[style=C]            
            Si no Entonces
                Agregar nodo a Secuencia
\end{lstlisting}
\end{tiny}



\begin{tiny}
\begin{lstlisting}[style=C]		
		Si no Entonces
            Borrar Secuencia
\end{lstlisting}
\end{tiny}


En la l\'inea 25, se presenta el caso contrario de la (l\'inea 4), que el \emph{Nodo} no exista en el grafo, este se agrega (l\'inea 7), enlaz\'andolo desde el \emph{Nodo} anterior.


\begin{tiny}
\begin{lstlisting}[style=C]
    Si no Entonces
        Colocar Nodo en grafo
Fin Mientras
Fin Procedimiento 
\end{lstlisting}
\end{tiny}



\begin{tiny}
\begin{lstlisting}[style=C]
Procedimiento Ejecucion
Mientras Verdad Hacer
\end{lstlisting}
\end{tiny}



\begin{tiny}
\begin{lstlisting}[style=C]
Escribir ``Selecciona la tarea a ejecutar''
Leer Seleccion
Escribir ``Ejecutar Accion seleccionada?''
Leer Decisi\'on
\end{lstlisting}
\end{tiny}



\begin{tiny}
\begin{lstlisting}[style=C]
Si Decision = ``Si'' Entonces
	Ejecutar Seleccion
\end{lstlisting}
\end{tiny}



\begin{tiny}
\begin{lstlisting}[style=C]
Fin Mientras
Fin Procedimiento
\end{lstlisting}
\end{tiny}

