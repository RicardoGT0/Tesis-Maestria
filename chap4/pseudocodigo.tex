
A continuaci\'on se presenta el pseudo--codigo del programa desarrollado.


De la l\'inea 1 a la 28 se define el inicio del \emph{Procedimiento de 
 monitoreo}, en este se leen los dispositivos de entrada, se genera el grafo 
 y se buscan las secuencias. Este proceso, es necesario que permanezca activo 
 desde que se enciende la computadora hasta que se apague el equipo, para 
 capturar cada acci\'on que realice el usuario, por lo que en la l\'inea 2 se 
 inicia un ciclo \emph{Mientras} o \emph{While} que no tiene condici\'on de 
 paro y abarca hasta la l\'inea 27. 

\begin{tiny}
\begin{lstlisting}[style=C]
Procedimiento monitoreo
Mientras Verdad Hacer
\end{lstlisting}
\end{tiny}

%\noindent
En la l\'inea 3, se lee la acci\'on del dispositivo de entrada y esta es 
 almacenada en forma de \emph{Nodo}.


\begin{tiny}
\begin{lstlisting}[style=C]
    Nodo = Capturar accion de teclado o raton
\end{lstlisting}
\end{tiny}

En la l\'inea 4, se verifica que exista un \emph{Nodo} con la misma 
 informaci\'on que el \emph{Nodo} reci\'en creado en el \emph{Grafo}, en caso 
 de ser as\'i, al \emph{Nodo} existente se le aumenta 1 al 
 \emph{contador\_Nodo}, operaci\'on realizada en la l\'inea 5, 
 adicionalmente, se va a empezar con una secuencia de validaciones que 
 servir\'an para identificar una secuencia y posteriormente una tarea. 

\begin{tiny}
\begin{lstlisting}[style=C]
    Si Nodo existe en Grafo Entonces
        Incrementar contador_Nodo
\end{lstlisting}
\end{tiny}

Primero, se verifica que el \emph{Nodo} se haya repetido mas de 70 veces.

\begin{tiny}
\begin{lstlisting}[style=C]
        Si contador_Nodo > 70 Entonces
\end{lstlisting}
\end{tiny}

En el caso de ser verdadera la verificaci\'on de la l\'inea 6, se verifica 
 que el \emph{Nodo} exista en una lista de nodos llamada \emph{Secuencia}, 
 esta es una tarea que ha realizado el usuario.

\begin{tiny}
\begin{lstlisting}[style=C]
            Si Nodo existe en Secuencia Entonces
\end{lstlisting}
\end{tiny}

En el caso de ser verdadera la verificaci\'on de la l\'inea 7, se verifica 
 que la \emph{Secuencia} exista en la \emph{lista\_Secuencias}.

\begin{tiny}
\begin{lstlisting}[style=C]
                Si Secuencia existe en lista_Secuencias Entonces
\end{lstlisting}
\end{tiny}

Si la verificaci\'on de la l\'inea 7 es verdadera, se verifica que la 
 \emph{Secuencia} se haya repetido mas de 5 veces, para validar que le pueda 
 ser de utilidad al usuario, adem\'as se verifica que la \emph{secuencia} no 
 exista en la \emph{lista\_Tareas} o en la \emph{lista\_Ignoradas}, ya que 
 esto significar\'ia que ha sido encontrada anteriormente. 
 En caso de que esta validaci\'on de la l\'inea 9 resulte verdadera
 y que el usuario considere que s\'i le es \'util, se le solicita 
 que le asigne un nombre a la \emph{Secuencia} mostrada. Posteriormente en la 
 l\'inea 11 se lee la \emph{Respuesta} introducida por el usuario.

\begin{tiny}
\begin{lstlisting}[style=C]
                    Si contador_Secuencia > 5 Y Secuencia No existe en lista_Tareas O en lista_Ignoradas Entonces
                        Escribir ``Si desea Guardar la tarea [Secuencia], escriba un nombre''
                        Leer Respuesta
\end{lstlisting}
\end{tiny}

En caso de que la \emph{respuesta} sea un nombre para la acci\'on se ejecuta 
 la l\'inea 14, en la que se agrega la \emph{Secuencia} a una 
 \emph{lista\_Tareas}, que son las \emph{Tareas} a las que tiene acceso el 
 usuario, para poder ejecutarlas. En caso contrario se prosigue con la 
 l\'inea 15, agregando la \emph{Secuencia} a una \emph{lista\_Ignoradas}, 
 estas son las tareas que el usuario ha decidido que no le son de utilidad y 
 por tal no se le volver\'an a mostrar.

\begin{tiny}
\begin{lstlisting}[style=C]  
                        Si Respuesta es nombre Entonces
                            Agregar Secuencia a lista_Tareas
                        Si no Entonces 
                            Agregar Secuencia a lista_Ignoradas
\end{lstlisting}
\end{tiny}

En caso contrario de que la verificaci\'on de la l\'inea 9 no sea verdadera, 
 en la l\'inea 17, se incrementa en 1 el \emph{contador\_Secuencia}, para 
 indicar que se ha usado una vez m\'as la \emph{Secuencia}.

\begin{tiny}
\begin{lstlisting}[style=C]
                    Si no Entonces
                        Incrementar contador_Secuencia
\end{lstlisting}
\end{tiny}

Para el caso de que la verificaci\'on de la l\'inea 8 sea falsa, se prosigue 
 con la l\'inea 19, agregando la \emph{Secuencia} a la 
 \emph{lista\_Secuencias}.

\begin{tiny}
\begin{lstlisting}[style=C]
                Si no Entonces
                    Agregar Secuencia a Lista_Secuencias
\end{lstlisting}
\end{tiny}

La instrucci\'on 20 esta contenida en el caso verdadero de la verificaci\'on 
 de la l\'inea 7, por tal, s\'i el \emph{Nodo} existe en la \emph{Secuencia}, 
 despu\'es de verificar si es una \emph{Tarea} o no, se \emph{Borra} la 
 \emph{Secuencia} dej\'andola sin \emph{Nodos}.

\begin{tiny}
\begin{lstlisting}[style=C]            
            Borrar Secuencia
\end{lstlisting}
\end{tiny}

La instrucci\'on 21 es el caso contrario de la verificaci\'on de la línea 7, 
 por tal, en caso de que el \emph{Nodo} no se encuentre en la 
 \emph{Secuencia}, este se agregar\'a.

\begin{tiny}
\begin{lstlisting}[style=C]            
            Si no Entonces
               Agregar nodo a Secuencia
\end{lstlisting}
\end{tiny}

La l\'inea 23 es para el caso falso de la verificaci\'on de la l\'inea 6, 
 como el \emph{Nodo} no cumple la condici\'on de tener mas de 70 
 repeticiones, se \emph{Borra} la \emph{Secuencia}.

\begin{tiny}
\begin{lstlisting}[style=C]		
        Si no Entonces
            Borrar Secuencia
\end{lstlisting}
\end{tiny}


En la l\'inea 25, se presenta el caso contrario de la (l\'inea 4), que el 
 \emph{Nodo} no exista en el grafo, este se agrega (l\'inea 26) en el 
 \emph{Grafo}, enlaz\'andolo desde el \emph{Nodo} anterior. 


\begin{tiny}
\begin{lstlisting}[style=C]
    Si no Entonces
        Agregar Nodo en Grafo
\end{lstlisting}
\end{tiny}


Por ultimo; en la l\'inea 27 se encuentra el final del \emph{Mientras} que se 
 inici\'o en la l\'inea 2, y en la l\'inea 28 est\'a el final del 
 \emph{Procedimiento monitoreo} que se empez\'o en la l\'inea 1.

\begin{tiny}
\begin{lstlisting}[style=C]
Fin Mientras
Fin Procedimiento 
\end{lstlisting}
\end{tiny}
